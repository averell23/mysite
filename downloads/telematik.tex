\documentclass[a4paper]{danarticle}
\usepackage{a4}
\usepackage{german}
\pagestyle{headings}
\sloppy


\begin{document}
  \author{Daniel Hahn}
  \title{Telematik on a Spickzettel}
  \maketitle
  
  \begin{abstract}
    Dies ist eine Zusammenfassung der verschiedenen Telematik-Inhalte die ich
    als Vorbereitung auf meine Pr"ufung bei Prof. Zitterbart im SS 01
    zusammengeschrieben habe. Die Erkl"arungen in diesem Dokument m"ussen nicht 
    unbedingt richtig sein,
    ich habe das ganze zusammengeschrieben um es selber besser zu verstehen. Du
    wurdest gewarnt. Au"serdem sind noch viele Teile unvollst"andig, z.B. gibt
    es zu ISDN viel mehr zu sagen als hier steht - und ich werde das Dokument
    auch nicht fortschreiben
    
    \textbf{Nochmal: Dieses Werk wurde zu "Ubungszwecken angefertigt. Es ist
    teilweise fehlerhaft und unvollst"andig. Es ist NICHT geeignet als
    alleiniges Lehrmaterial f"ur eine Pr"ufung. Du wurdest gewarnt.}
    
  \end{abstract}
  
  \section*{Die sieben Schichten}
    Die sieben Schichten stammen aus dem ISO/OSI Basis-Referenzmodell.
    Eigentlich sind es nur sieben, weil IBM damals sieben hatte, und das ganze
    Modell ist leicht konfus. Das Modell geht davon aus, da"s jede Schicht ihre
    Dienste an Dienstzugangspunkten (SAP - Service Access Point) der
    n"achsth"oheren Schicht zur Verf"ugung stellt.
  \section{Bit"ubertrangsschicht}
    Auch: Physical Layer. Diese Schicht befasst sich mit dem Transpor
    einzelner Bits zwischen zwei Ger"aten.
    \subsection*{Theoretische Grundlagen}
      \subsubsection*{Fourier-Analyse}
        M. Fourier entdeckte, da"s jede jede periodische Funktion $g(t)$ sich
        durch die Summe (unendlich) vieler Sinus- und Cosinusfunktionen
        nachbilden l"asst.
        \[
          g(t) = \sum^{\infty}_{n=1} a_n \sin(2 \pi nft) +
                 \sum^{\infty}_{n=1} b_n \cos(2 \pi nft)
        \]
        Dabei ist $ f = 1/T $ die Grundfrequenz und $ a_n $ und $ b_n $ die
        Amplituden der $n$-ten Harmonischen.
          
        \textbf{Bedeutung:} Eine Funktion (also z.B. ein Datensignal) 
        l"asst sich als
        Kombination (m"oglicherweise unendlich) vieler Sinus- und
        Cosinusschwingungen verschiedener Frequenz auffassen. Ist der
        Frequenzraum beschr"ankt (durch D"ampfung oder per Filter), so k"onnen
        einige der Sinus- und Cossinusschwingungen nicht passieren, und das
        Ausganssignal wird verf"alscht.
      \subsubsection*{Nyquist-Theorem}
        Wenn ein Signal eine Bandbreite von $ H $ Hertz zur Verf"ugung hat,
        kann es durch $ 2H $ Samples pro Sekunde vollst"andig wieder
        hergestellt werden. Eine h"ohere Abtastrate bringt keinen Vorteil, da
        ja keine h"oheren Frequenzen vorhanden sind, die durch die h"ohere 
        Abtastrate entdeckt werden k"onnten.
          
        \textbf{Dumme Frage:} Warum $ 2H $ und nicht $ H $? (Vielleicht weil
        nur so jede wichtige "Anderung \glq erwischt\grq wird?
          
        Dies begrenzt die maximale Datenrate auf einem (rauschfreien!) Kanal
        auf 
        \[ 2H \log_2 V \mbox{\hspace{1mm} Bit/Sekunde} \]
        wenn pro Abtastung $ V $ diskrete Signale unterschieden werden
        k"onnen. (Logischerweise kann die Kapazit"at hier immer noch beliebig
        gro"s sein, wenn man $ V $ entsprechend w"ahlt).
      \subsubsection*{Shannon-Theorem}
        Shannon hat gezeigt, da"s in einem verrauschten Kanal maximal
        \[ H log_2 (1 + S/N) \]
        Bit pro Sekunde "ubertragen k"onnen (egal mit welcher Technik). Im
        Vergleich zu Nyquist: Es ist nicht m"oglich, beliebig viele diskrete
        Stufen in einem verrauschten Kanal zu unterscheiden.
        
        \textbf{Rauschabstand:} $ (S/N) $ ist im vorigen Beispiel der
        Rauschabstand, d.h. das Verh"altnis zwischen Signal und Rauschen. $ S
        $ ist hier die Signalst"arke, und $ N $ die St"arke des Rauschens --
        je h"oher dieser ist, desto besser f"ur die "Ubertragung. Der
        Rauschabstand wird auch in dB (Dezibel angegeben), diese Einheit ist
        bezeichnet den Wert von $ 10 \log_10 (S/N) $. (D.h. RA 10 = 10 db, RA
        100 = 20 dB, RA 1000 = 30 dB)
    \subsection*{"Ubertragungstechniken}
      \subsubsection*{Medien}
        \begin{itemize}
          \item{Massenspeichermedien}
          \item{Vedrilltes Kabelpaar/Twisted Pair (Vedrillung hilf gegen
          St"orungen}
          \item{Koaxialkabel, Basisband}
          \item{Koaxialkabel, Breitband (mehrere Kan"ale, mit Verst"arkung)}
          \item{Lichtwellenleiter/Glasfaser (s. unten)}
          \item{Infrarot}
          \item{Funk (Radio/Mikrowelle)}
          \item{Satellit}
          \item{Laser}
        \end{itemize}
        \textbf{Glasfasern} haben eine besonders hohe
        "Ubertragungsrate, sie k"onnen Multimode (mehrere Frequenzen in der
        Faser, billige LED-Technik) oder Single-Mode (nur eine Frequenz in der
        Faser, teuer, nur Laser, bessere "Ubertragungsqualit"at sein.
    \subsection*{Besonderheiten bei der (drahtlosen) "Ubertragung}
      \subsubsection*{Frequenzband}
        Radio"ubertragung: LF 148, 5 bis 283,5 kHz, MF 520 kHz bis 1605,5 kHz,
        SW 5,9 MHz bis 26,1 MHz, UKW 87,5 bis 108 MHz. 
       
        Interessant: DAB 223-230, 1452-1472 MHz, dig. Fernsehen 470-862 MHz.
        Mobiltelefon: Analog: 450-465 MHz, GSM 890-960 MHz, 1710-1880 MHz u.a.
        (ca. 500 MHz - 2GHz). Schnurlose Telfone 1-2 GHz (DECT: 1880-1900 MHz).
        IEEE 802.11: 2.4 GHz
        
        \textbf{Generell gilt:} Je h"oher die Frequenz, desto eher verh"alt sich
        die Sendung wie Licht, und desto besser f"ur die Daten"ubertragung.
        
        \textbf{Dumme Frage:} Warum sind h"ohere Frequenzen besser?
      \subsubsection*{Signalausbreitung}Eine \glqq
        Funkzelle\grqq\ um einen Sender besteht aus dem "Ubertragungsbereich, in
        dem kommuniziert werden kann, dem Erkennungsbereich, in dem eine Sendung
        erkannt, aber nicht verstanden werden kann, und dem Interferenzbereich,
        in dem eine Sendung nur noch als Hintergrundrauschen ist. Die Ausdehnung
        dieser Bereich h"angt von verschiedenen Parametern ab.
        
        \textbf{D"ampfung:} Im Vakuum nimmt die Leistung des Signals im Quadrat
        der Entfernung ab. Im wirklichen Leben wird die Ausbreitung au"serdem
        noch von Athmosph"arischen- und Gel"andebedingungen beeinflu"st. Je
        h"oher die Sendefrequenz, umso st"arker werden die Wellen von
        Hindernissen absorbiert, bis hin zur v"olligen Abschattung.
        
        \textbf{Reflexion, Streuung, Beugung:} Ist ein Hindernis im Vergleich 
        wesentlich gr"o"ser als die Wellenl"ange, so kann es die Wellen
        reflektieren. An kleinen Objekten kommt es dagegen zur Streuung der
        Wellen. Zus"atzlich kann an Kanten von Hindernissen eine Beugung
        auftreten.
      \subsubsection*{Mehrwegeausbreitung}
        F"ur ein Signal (durch Streuung und Reflexion) viele verschieden lange
        Ausbreitungspfade bis zum Empf"anger, und die Signale dieser Pfade gehen
        zu verschiedenen Zeiten dort ein. Dadurch kann einerseits das gerade
        gesendet Symbol \glqq verschmiert\grqq\ werden, und es kann zu "Uberlappungen mit
        den nachfolgenden Daten kommen (ISI, Intersymbolinterferenz)
        
        Damit das System dies kompensieren kann, kann der Sender eine bekannte
        Trainingssequenz senden, und der Empf"anger kann darauf seinen Entzerrer
        entsprechend programmieren.
      \subsubsection*{Regulierung der Frequenzen}
        Die Frequenzen werden meist durch nationale Beh"orden (USA: FCC, Europa:
        CEPT)
        vergeben. Internationale Koordinierung durch ITU (International
        Telecommunications Union) in Genf. (Unterteilung nach Region 1: Europa,
        Mittlerer Osten, Afrika, Region 2: Amerika, Gr"onland, Region 3: Ferner
        Osten, Australien, Neuseeland)
    \subsection*{Modulationsverfahren}
      \subsubsection*{Basisbandmodulation}
        Das einfachste: Strom/kein Strom. Abgesehen von den Gleichstromanteilen
        ist es f"ur die Funk"ubertragung auch nicht besonders geeignet.
      \subsubsection*{Amplitudenmodulation}
        Die Amplitude der Schwingungen (z.B. Sinusschwingungen) wird variiert.
        Einfachster Fall: Sinus/kein Sinus. F"ur Glasfaser"ubertragung geht das
        (Licht/kein Licht), im Funkverkehr wird diese Methode stark von
        St"orungen beeinflusst.
      \subsubsection*{Frequenzmodulation}
        Die Frequenz der Schwingung wird variiert, um Daten zu "ubertragen (z.B.
        niedrige Frequenz/hohe Frequenz). Spezielle Modulatoren k"onnen
        Phasenspr"unge verhindern (Continuous Phase Modulation, CPM). Bei
        Phasenspr"ungen w"urden unerw"unscht hohe Frequenzen auftreten.
      \subsubsection*{Phasenmodulation}
        Die Phase der Sinusschwingung wird verschoben um Daten zu codieren.
        (z.B. Sinus normal, 180 Grad verschoben). Dieses Verfahren wird
        besonders bei der Modemkommunikation gern angewendet.
      \subsubsection*{Kombinationen}
        Diese Verfahren k"onnen fast beliebig kombiniert werden, z.B. bei Modems
        gerne Phasen- plus Amplitudenmodulation.
      \subsubsection*{Manchester-Codierung}
        Codierung f"ur Kabel"ubertragung. Zwar soll auch basisbandm"a"sig
        Strom/kein Strom verwendet werden, allerdings sollen Gleichstromteile
        vermieden werden. Deshalb gibt es in jedem Bit auf jeden Fall einen
        "Ubergang: high->low bedeutet 1, low->high bedeutet 0 (normales
        Manchester. Differenzielle Manchester: Anfang des Intervalls ist
        wichtig: Ein "Ubergang bedeutet 0, kein "Ubergang bedeutet 1. 
        Nachteil dieser Methode: Die "Ubertragungsrate wird halbiert (da
        doppelte Frequenz ben"otigt wird).
    \subsection*{Multiplexverfahren} 
      \subsubsection*{Raummultiplex}
        Niemand merkt, was auf der anderen Seite des Zimmers passiert -- oder in
        der n"achsten Funkzelle.
      \subsubsection*{Frequenzmultiplex}
        Eine Frequenz pro "Ubertragung. Funktioniert prima einfach, allerdings
        brauchen wir einen Schutzabstand zwischen den Frequenzen. Eignet sich
        haupts"achlich f"ur dauerhaft ben"otigte Kan"ale.
      \subsubsection*{Zeitmultiplex}
        Jeder kommt abwechselnd dran. Die Bandbreite kann nach Bedarf neu
        verteilt werden, allerdings ist Synchronisation erforderlich.
      \subsubsection*{Kombiverfahren, Channel Hopping}
        Alle Verfahren k"onnen nat"urlich auch kombiniert werden. Bei der
        Kanalspringerei wird hat jeder Kanal bestimmte Zeitschlitze auf
        verschiedenen Kan"alen. Das gibt es in langsam (Frequenz wird alle paar
        Bit gewechselt) und schnell (Frequenz wird pro Bit mehrfach gewechselt).
        Beispiel f"ur schnellen Wechsel: Bluetooth, (GSM optional). Beispiel
        f"ur langsamen Wechsel: GSM.
      \subsubsection*{Codemultiplex}
        Etwas kompliziert: Alle senden gleichzeitig, und trotzdem klappt es:
        Jede Station hat eine sogenannte Chippingsequenz (typischerweise 64 oder
        128 bit, oder 1000 bei der Army). Die Sequenzen der Stationen sollten
        orthogonal zueinander sein.
        
        Bei der "Ubertragung hat ein Bit genau die L"ange der Chippingsequenz
        und wird mit dieser per XOR verkn"upft (bzw.: F"ur eine 0 wird die
        Chipfolge "ubertragen, f"ur eine 1 ihr Komplement (oder auch umgekehrt).
        Dadurch wird die ben"otigte Bandbreite multipliziert, allerdings k"onnen
        alle Stationen im selben Frequenzbereich "ubertragen.
        
        Der Empf"anger mu"s mit dem Sender exakt synchronisiert sein, beim
        Empfang wird das Produkt aus dem empfangenen Signal und der Chipfolge
        gebildet. Signale anderer Sender werden (da sie orthogonal sind)
        \glqq herausgek"urzt\grqq . Einzelne falsch empfangene Chips werden
        ausgeglichen, da das Produkt durch einzelne Fehler nur leicht
        verf"alscht wird.
    \subsection*{PCM und SONET}
      PCM: Standardverfahren f"ur die Telefonie. Es werden 8k Abtastungen pro
      Sekunde gemacht, laut Nyquist genug f"ur einen 4kHz-Kanal. Jede Abtastung
      liefert 7 Bit, eine Abtastung erfolgt alle 125 $\mu$S. Zeitintervalle in
      TK-Einrichtungen sind daher meist Vielfache von 125 $\mu$S.
      
      Die Telefongesellschaften benutzen als kleinste Einheit den
      \textbf{T1-Tr"ager}. Es werden 24 Kan"ale "ubertragen, und zwar in
      Einheiten von 193 Bit/s (1,544 MBit/s). Jeder Kanal hat pro Block 8 Bit, 7
      f"ur Daten, 1 Bit f"ur Steuerung. 1 Bit dient der Rahmenbildung
      (abwechselnd 0/1). Die CCITT-Empfehlung schl"agt vor, 8 Bit f"ur Daten zz
      verwenden. Entweder wird ein zus"atzliches Zeichengabebit am Rahmen
      angeh"angt, oder jeder 6 Sample entha"lt ein Zeichengabebit (d.h. jeder 6
      Sample enth"alt nur 7 Bit Daten). Die 24 Kan"ale werden meist reihum
      abgetastet. 
      
      Man kann in PCM-Kan"alen zus"atzlich differenzielle oder pr"adiktive
      Codierung verwenden (codiere Abstand zum vorherigen und vorhergesagten
      Wert), um die Bitmenge zu verkleiner.
      
      \textbf{SONET} (Syncronous Optical Network) ist ein Standard f"ur optische
      "Ubertragungsnetze. SONET besteht aus einer kompletten Architektur:
      Multiplexer und Repeater. Die Daten werden "uber einen \textit{Pfad} vom
      Start bis zum Ziel geleitet. Die Strecke dazwischen besteht aus mehreren
      \textit{Leitungen} (zwischen verschiedenen Multiplexern, die aus
      \textit{Abschnitten} (zwischen Multiplexern und Repeatern) bestehen. SONET
      wird von einem Mastertakt einer Genauigkeit von $ 10^9 $ gesteuert.
      
      Ein SONET-Block besteht aus 810 byte und es werde 8000 pro Sekunde
      "ubertragen. Die 810 byte k"onnen in einer 9 $ \times $ 90 Tabelle
      dargestellt werden. Die ersten 3 Spalten sind f"ur Leitungs- und
      Abschnittsoverhead reserviert (3 Zeilen Leitung, 6 Zeilen Abschnitt). Ein
      SPE-Block (Synchronous Payload Envelope) ist 87 byte lang, das erste Byte
      enth"alt Pfadoverhead. Er kann an einer beliebigen Stelle im Rahmen
      beginnen, das erste Byte im Abschnittsoverhead enth"alt einen Zeiger auf
      den Anfang. Die Datenrate betr"agt 51,84 Mbps brutto, und 50,112 Mbps
      netto.
    \subsection*{Repeater}
      Ein Repeater verbindet Netzwerke auf der Bit"ubertragungschicht. Er ist
      zwar aufwendiger als ein reiner Verst"arker, indem er digitale Signale
      decodiert und reproduziert, wei"s jedoch nichts von der Semantik der
      Daten.
  \section{Sicherungsschicht}
    Die Sicherungsschicht hat die Aufgabe, eine zuverl"assige (gesicherte)
    Verbindung zwischen zwei benachbarten Kommunikationspartnern herzustellen.
    
    Die Sicherungsschicht kann der Vermittlungsschicht verschiedene Dienstg"uten
    anbieten: Verbindungslos (best"atigt oder unbest"atigt) und
    verbindungsorientiert.
    \subsection*{Rahmenbildung}
      Die Vermittlungsschicht mu"s in jedem Fall die Daten irgendwie verpacken.
      Daher ist es notwendig, den kontinuierlichen Bitstrom, den Schicht 1 zur
      Verf"ugung stellt, aufzuteilen. Es gibt mehrer M"oglichkeiten, Rahmen zu
      erzeugen:
      
      Am simpelsten ist die Methode, die eingegangenen Bytes abzuz"ahlen,
      m"oglicherweise wird am Anfang des Rahmens jeweils noch dessen L"ange
      eingef"ugt. Diese Methode versagt jedoch bei Byteverlusten, oder wenn die
      L"ange des Rahmens falsch "ubertragen wird. Besser ist es, Anfang und Ende
      eines Rahmens irgendwie zu markieren: Beim 
      \textbf{Zeichenstopfen} wird Anfang und
      Ende des Rahmens durch eine bestimmte Zeichenkombination markiert. Kommt
      ein Markierungszeichen in den Daten vor, so wird es \glqq escaped \grqq\
      (z.B. verdoppelt). Die Escape-Codes werden dann beim Empf"anger wieder
      entfernt. Beim \textbf{Bitstopfen} wird eine bestimmte Bitkombination als
      Zeichengrenze eingef"ugt. Kommt diese Kombination in den Daten vor, so
      wird ein zus"atzliches Bit \glqq eingestopft \grqq , das beim Empf"anger
      wieder entfernt wird.
      
      Eine weitere Methode, Rahmen zu bilden ist die \textbf{Coderegeln der
      Bit"ubertragungsschicht zu verletzen}. z.B. w"urde bei der
      Manchester-Codierung in der Mitte eines Bit kein "Ubergang stattfinden.
    \subsection*{Fehlererkennung/-korrektur}
      Fehlererkennung und -korrektur beruht auf dem absichtlichen Einf"ugen von
      Redundanz in die Daten.
      \subsubsection*{Hamming-Abstand}
        Die Hamming-Distanz zweier Bitfolgen ist die Zahl der Bits, die
        ge"andert werden m"ussen um die eine Folge in die andere zu verwandeln.
        Dieser Abstand gibt auch an, wie viele Bits ver"andert werden m"ussen,
        damit die beiden Folgen nicht mehr unterscheidbar sind.
      \subsection*{Parit"atsbit}
        Sollte klar sein.
      \subsubsection*{Hamming-Code}
        Damit n-bit-Fehler erkannt werden k"onnen m"u"s die Hamming-Distanz
        mindestens $ n + 1 $ betragen (damit der Abstand zum n"achsten g"ultigen
        Codewort gr"o"ser ist, als der Fehler. Entsprechend mu"s f"ur n-bit
        Fehlererkennung der Abstand mindestens $ 2n + 1 $ betragen.
        
        Der Hammingcode kann 1-bit-Fehler mit einer minimalen Anzahl von
        Pr"ufbits korrigieren. Dazu werden alle bits Positionen einer
        Zweierpotenz liegen Pr"ufbits. Die Pr"ufbits enthalten die Parit"at
        aller bits, die die entsprechende Zweierpotenz in ihrer Zerlegung
        enthalten (Bsp. 17 -> Pr"ufbits 16 + 1).
        
        Bei einem Bitfehler werden die Werte aller Pr"ufbits, die falsch sind,
        addiert um die Position des falschen Bits zu erhalten.
      \subsubsection*{CRC Pr"ufsummen}
        Meist m"ochte man Fehler nicht korrigieren (zu hoher Overhead), sondern
        lediglich erkennen. Beim CRC werden die Daten als Polinom aufgefasst,
        das die Koeffizienten 0 und 1 hat. Au"serdem wird ein Pr"ufpolinom 
        $ G(x) $ vereinbart, das beiden Parteien bekannt sein mu"s. Hat das
        Pr"ufpolynom den Grad $ r $, so werden $ r $ Pr"ufbits angeh"angt. Diese
        enthalten den Rest der Division des Datenpolynoms und $ G(x) $. 
        Das Ergebnis l"a"st sich ohne Rest durch $ G(x) $ teilen, bleibt ein
        Rest bei der "Uberpr"ufung, liegt ein Fehler vor.
    \subsection*{Flusskontrolle}
      Die Flusskontrolle soll verhindern, da"s der Empf"anger von Daten
      "uberschwemmt wird, die er nicht annehmen kann. Au"serdem sollen diese
      Protokolle den Verlust und die Duplizierung von Rahmen verhindern.
      \subsubsection*{Stop-and-Wait}
        Ganz einfach: Der Sender wartet vor jedem neuen Rahmen auf die
        Best"atigung des vorhergehenden. Es wird eine 1-Bit Sequenznummer
        verwendet, um den Verlust eines Rahmens (bzw. einer Best"atigung) 
        zu erkennen. F"ur den Fall, das etwas h"angt, gibt es Timeouts.
      \subsubsection*{Sliding Window (Schiebefenster)}
        Sender und Empf"anger haben jeweils ein Fenster von Paketen, die
        ausstehen d"urfen. Ein Sliding Window der Gr"osse 1 entspricht
        Stop-and-Wait. Alle Rahmen haben Sequenznummern, ein Fenster enspricht
        einem Sequenznummernbereich.
        
        Hat der Sender ein Fenster von gr"osser 1, so kann er ohne Best"atigung
        Rahmen senden, bis er die Grenze des Fensters erreicht hat. (Alle Rahmen
        im Fenster m"ussen gepuffert werden). Erh"alt er eine Best"atigung f"ur
        den ersten Rahmen, so wird das Fenster um eine Position weitergeschoben.
        
        Hat der Empf"anger ein Fenster gr"osser 1, so kann er Rahmen puffern,
        die ausser der Reihe ankommen. Ein Rahmen wird nur best"atigt, wenn alle
        vorherigen Rahmen korrekt angekommen sind. Sobald der unterste Rahmen
        best"atigt wurde, wird das Fenster um eine Position weitergeschoben.
        Rahmen, die au"serhalb des Fensters ankommen, werden verworfen.
      \subsubsection*{Wiederholung von fehlerhaften Rahmen}
        Wird ein Rahmen vom Empf"anger als fehlerhaft erkannt, oder geht er
        verloren, so mu"s er nochmals gesendet werden. Beim
        \textbf{Go-Back-N}-Verfahren werden, wenn der Timeout f"ur einen Rahmen
        abl"auft, dieser Rahmen und alle nachfolgenden erneut gesendet, d.h.
        alle Sendungen seit dem Fehler werden wiederholt. Dies entspricht einem
        Empfangsfenster von 1. Falls der Empf"anger einen Puffer hat, kann
        stattdessen \textbf{Selective Repeat} verwendet werden. In diesem Fall
        wird nur der fehlerhafte Rahmen neu "ubertragen, alle anderen hat der
        Empf"anger noch in seinem Puffer.
        
        Eine Optimierung ist die NACK-Nachricht, mit der der Empf"anger den
        Sender anweist, einen bestimmten Rahmen nochmals zu senden.
      \subsubsection*{Piggybacking}
        Best"atigungen k"onnen auch mit Datenpaketen in die Gegenrichtung \glqq
        mitverschickt \grqq\ werden, allerdings mu"s man abw"agen, wie lange die
        Best"atigung auf ein Datenpaket warten soll, bevor sie alleine
        losgeschickt wird.
    \subsection{Sicherungsschicht-Protokolle}
      \subsubsection*{HDLC}
        HDLC (High Level Data Link Control) ist ein Protokoll der
        Sicherungsschicht, das in ziemlich vielen verschiedenen Variationen
        vorkommt. Andere Namen sind LAP und LAPD.
        
        Diese Protokolle sind bitorientiert, ein Rahmen wird durch die
        Flagsequenz 01111110 an Anfang und Ende begrenzt. Die Rahmen enthalten
        ein ein Byte grosses Adressfeld und ein ebenso grosses Steuerungsfeld.
        Die L"ange der "Ubertragenen Daten ist variabel. Das gesamte Paket wird
        durch eine 2-byte Pr"ufsumme gesch"utzt (CRC-Variante, die fehlerhafte
        Flagbits erkennt).
        
        HDLC kennt drei Rahmentypen: Informations-, Steuerungs- und
        unnummerierte Rahmen. Informationsrahmen enthalten eine Sequenznummer
        und ein ACK-Nummer (die angibt, welche Sequenznummer von der Gegenstelle
        als n"achstes erwartet wird). Ein P/F-Bit (Poll/Final) war
        haupts"achlich f"ur pollende Terminalapplikationen gedacht. 
        
        Die Steuerungrahmen k"onnen verschiedene Typen haben, z.B. Best"atigung
        (ACKNOWLEDGEMENT), Zur"uckweisung (REJECT, f"ur go-back-n), selektive
        Zur"uckweisung (SELECTIVE REJECT, fordert selektive Wiederholung, u.�).
        
        Unnummerierte Rahmen enthalten Steuerinformationen, wie z.B.
        Verbindungsabbauw"unsche (DISC, Disconnect). Au"serdem gibt es einen UA
        (Unnumbered Acknowledgement) Rahmen, der den letzten ausstehenden
        Steuerrahmen best"atigt (Steuerrahmen haben keine Sequenznummer, es darf
        immer nur ein Steuerrahmen ausstehen).
      \subsubsection*{SLIP}
        SLIP ist ein recht primitives Protokoll zur "Ubertragung von IP-Paketen
        "uber serielle Leitungen (daher auch Serial Line IP). Die IP-Pakete
        werden einfach in Rahmen eingebaut, die durch ein Flagbyte gebildet
        werden. Das Flagbyte wird in den Daten durch eine seltsame Escapesequenz
        umgangen, so da"s das erste Byte der Escapesequenz selbst wieder
        gestopft werden mu"s...
        Optimierungen in SLIP betreffen das Weglassen mehrfach vorhandener
        Header und das inkrementelle "ubertragen Headerfeldern. 
        
        SLIP ist kein Internetstandard und in teilweise inkompatiblen Versionen
        vorhanden. Es unterst"utzt nur IP, keine dynamischen Adressen, keine
        Authentifiezierung und keinerlei Fehlerkontrolle. SLIP wird im
        Augenblick an fast allen Stellen von PPP abgel"ost.
      \subsubsection*{PPP}
        PPP ist ein recht leistungsf"ahiges Protokolle f"ur
        Punkt-zu-Punkt-Verbidungen. Die Verbindung kann authentifieziert
        aufgebaut werden, es k"onnen fast beliebige Protokolle "uber PPP
        "ubertragen werden und die Partner k"onnen die Parameter der Verbindung
        dynamisch aushandeln.
        
        PPP ist zeichenorientiert, das Rahmenformat ist allerdings "ahnlich dem
        HDLC-Rahmenformat. Auch PPP verwendet das Flagbyte 01111110, es wird
        allerdings zeichengestopft, wenn es in den Daten vorkommt. Das
        Adressfeld (11111111) und das Steuerungsfeld (00000011) haben im
        allgemeinen feste Werte, es kann vereinbart werden, sie wegzulassen.
        Danach folgen zwei (optional ein) Byte, die das verwendete Protokoll angeben (NCP, LCP,
        AppleTalk, IP...). Die Daten haben eine variable L"ange (bis zu einem
        vereibarten Maximum), die Pr"ufsumme
        hat zwei oder vier Byte.
        
        \textbf{LCP}, das Link Control Protocol, dient haupts"achlich zum Auf-
        und Abbau von PPP-Verbindungen. Mit dem LCP k"onnen
        verbindunsspezifische Parameter (Authetifikation, Art der unterst"utzten
        Protokolle, maximale Datenl"ange...) ausgehandelt werden.
        
        \textbf{NCP}, das Network Control Protocol, ist spezifisch f"ur jeden
        unterst"utzten Protokolltyp (z.B. IP oder AppleTalk). Es dient dazu,
        Parameter f"ur das "uber PPP betriebene Protokoll auszuhandeln. Bei IP
        w"aren dies z.B. die IP-Adressen und die Adressen von Gateways.
  \section*{Sicherungsschicht: MAC Teilschicht}
    Die MAC (Medium Access Control) ist ein Teil der Sicherrungsschicht, und
    hat die Aufgabe, den Zugriff mehrerer Kommunikationspartner auf ein
    einzelnes Medium zu koordinieren.
    \subsection*{ALOHA}
      Einfach und geradeaus: Jeder sendet wann es ihm passt, wenn es eine
      Kollision gibt: Pech gehabt. Das ganze hat eine maximale Effizienz von 10
      Prozent. Wenn man immerhin Zeitschlitze verwendet (Slotted Aloha)
      verdoppelt sich die Effizienz - allerdings ist dann eine Synchronisation
      aller Stationen notwendig.
    \subsection*{CSMA}
      Etwas schlauer ist es, vor dem Senden das Medium abzuh"oren, um zu sehen
      ob schon jemand sendet. Im einfachsten Fall (1-persistent CDMA) wird
      sofort gesendet, wenn das Medium frei ist (Wahrscheinlichkeit 1). Wenn
      eine Kollision auftritt, wartet jede Station eine zuf"allige Zeit, bevor
      sie es noch einmal versucht. Wenn mehrere Stationen auf das Medium warten,
      tritt auf jeden Fall eine Kollision auf. Um das zu verhindern, kann
      etweder die Wahrscheinlichkeit gesenkt werden, mit der die Station sendet
      (z.B. 0,5 das gesendet wird, und 0,5 das noch einen Schlitz abgewartet
      wird -> 0.5-persistent), oder die Station wartet eine zuf"allige Zeit,
      wenn das Medium belegt ist, und probiert es dann noch einmal
      (nonpersistent). Dieses Verfahren hei"st Carrier Sense Multiple Access.
    \subsection*{CSMA/CD}
      Noch besser kann der Zugriff auf das Medium geregelt werden, wenn die
      angeschlossenen Stationen eine Kollision sofort erkennen, und dann das
      Medium wieder freigeben. In diesem Fall kann, wenn eine Kollision erkannt
      wird, gleich ein neuer Versuch zur Belegung gestartet werden.
      
      Bei CSMA/CD besteht die "Ubertragung aus Perioden, in denen Daten
      "ubertragen werden und Konkurrenzperioden. W"ahrend einer
      Konkurrenzperiode versuchen verschiedene Stationen, das Medium zu belegen,
      und erst wenn keine Kollisionen mehr auftreten, beginnt die eigentliche
      Sendeperiode.
      
      Die Konkurrenzperiode kann man sich in Zeitschlitze unterteilt vorstellen,
      die der Zeit entsprechen, die zum Erkennen einer Kollision ben"otigt
      werden.
      \subsubsection*{Kollisionserkennung}
        Wenn ein Signal eine Zeit von $ \tau $
        ben"otigt, um ein Kabel der maximalen L"ange zu durchqueren, so kann eine
        Kollision in einer Zeit von $ 2\tau $ sicher erkannt werden. Ein
        Senderahmen mu"s mindestens so lang sein, da"s eine Kollision
        einwandfrei erkannt wird, d.h. er mu"s aus einer "Ubertragung der L"ange
        $ 2\tau $ bestehen.
        
        Ein konkurrierender Zugriff wird daher nach ca. $ 2\tau $ abgebrochen.
        (Dies ist der Zeitschlitz f"ur die Konkurrenzperiode).
      \subsubsection*{Exponential Binary Backoff}
        Dies ist eine Methode, mit der der Zugriff auf ein Medium f"ur viele
        Stationen organisiert werden kann. Sobald das Medium nach einer Sendung
        wieder freigegeben wird, wartet eine Station zuf"allig 
        0 oder 1 Zeitschlitze bevor
        sie zu senden versucht. Kommt es zu einer Kollision, wird die Anzahl der
        Wartschlitze verdoppelt (d.h. es wird jetzt 0-3 Schlitze gewartet).
        Diese Verdoppelung tritt nach jeder Kollision ein, solange bis eine
        Station kollisionsfrei sendet, oder ein Maximalwert erreicht wird.
    \subsection*{RTS/CTS, versteckte und ausgelieferte Ger"ate}
      Kollisionserkennung l"a"st sich bei Funknetzen nicht durchf"uhren, da der
      Sender pr"uft, ob das Medium belegt ist, Kollisionen aber beim Empf"anger
      auftreten. Bei der Funk"ubertragung k"onnen allerdings nicht alle
      Stationen alle anderen Stationen empfangen.
      \subsubsection*{Versteckte Endger"ate}
        Die Station A sendet an die Station B. Station C liegt in Reichweite von
        B, nicht jedoch von A. Wenn nun C an B senden m"ochte pr"uft sie, ob das
        Medium belegt ist. Da sie die Signal von A nicht empfangen kann, beginnt
        sie zu senden. Bei B treten nun Kollisionen auf.
      \subsubsection*{Ausgelieferte Endger"ate}
        In der selben Situation m"ochte B an C senden, w"ahrend A mit einer
        anderen Station kommuniziert, die au"serhalb der Reichweite von B und C
        liegt. B findet das Medium belegt, und sendet nicht an C, obwohl die
        Sendung von A nicht st"oren w"urde.
      \subsubsection*{Request to send/Clear to send}
        Diese Probleme lassen sich mit einem einfachen Mechanismus l"osen: Bevor
        ein Rahmen gesendet wird, schickt die sendenden Station A ein RTS (Request
        To Send) Signal. Die Empfangsstation B best"atigt dieses Signal mit CTS
        (Clear To Send) und der Kennung der sendenden Station. Alle Stationen in
        Reichweite des Empf"angers wissen nun, das er ein Signal von A empf"angt
        und werden erst nach dem Ende dieser "Ubertragung wieder eine Verbindung
        versuchen. Wenn der RTS-Rahmen kollidiert wird nur eine relativ kurze
        "Ubertragung (der RTS-Rahmen verloren). Falls der RTS-Rahmen verlorgen
        geht (d.h. es wird kein CTS empfangen) wird ein Backoff-Verfahren
        angewendet.
        
        Dieses Verfahren erforder einen erh"ohten Overhead, so da"s es z.B. bei
        802.11 optional ist.
    \subsection*{Kollisionsfreie Verfahren u.a.}
      Es ist auch m"oglich, den Zugriff auf das Medium so aufzuteilen, da"s
      keine Kollisionen auftreten. Diese Verfahren sind allerdings nicht so
      verbreitet wie die Kollisionbehafteten.
      \subsubsection*{Bitmusterprotokoll}
        Jede Station hat in der \glqq Konkurrenzphase \grqq\ ein reserviertes
        Bit, welches gesetzt wird, wenn die Station "ubertragen m"ochte. Nach
        dieser Phase wissen alle Stationen Bescheid, wer senden m"ochte und jede
        Station darf der Reihe nach einen Rahmen senden.
      \subsubsection*{Bin"arer Countdown}
        Alle Stationen senden gleichzeitig ihre Adressen, dabei wird eine 0 von
        1-Bits "uberschrieben. Alle Stationen, die bemerken da"s ihr Adressbit
        "uberschrieben wurde, scheiden sofort aus (d.h. nur noch Stationen mit
        einer 1 an der aktuellen Stelle machen weiter).
        Am Ende gewinnt die Station mit der h"ochsten ID, die ihren Rahmen
        senden darf.
      \subsubsection*{Adaptive Tree Walk}
        Noch ein etwas esoterisches Protokoll: Es beruht darauf, das die
        Stationen in einem bin"aren Baum angeordnet sind. Wenn eine Kollision
        auftritt, scheidet eine Seite des Baumes aus, und das Protokoll wird mit
        dem rechten oder linken Unterbaum fortgesetzt.
    \subsection*{Bridging}
      Eine Bridge ist ein Ger"at, da"s zwischen zwei (oder mehr) Netzwerken auf
      Sicherungsschicht-Ebene Vermittelt. Eine Bridge kann evtl. auch Rahmen von
      Format des einen Netzwerk in ein anderes umwandeln (z.b 802.3 nach 802.5).
      \subsubsection*{Transparente Bridges}
        Eine transparente Bridge kann in ein Netzwerk eingef"ugt werden, ohne
        da"s die Stationen davon etwas mitbekommen. Die Bridge arbeitet ohne
        Eingriff und routet die Rahmen mit dem Backward-Learning-Verfahren (s.
        Routing/Vermittlungsschicht).
      \subsubsection*{Spanning Tree}
        Falls mehrere Bridges an ein LAN angeschlossen sind, verwenden sie den
        Spanning-Tree-Algorithmus um zu verhindern, da"s Schleifen entstehen.
        (Daf"ur tauschen die Bridges untereinander Informationen aus). Die
        Bridge mit der niedrigsten ID wird die Wurzel des Baumes.
        
        Jede Bridge kennt die Kosten eines ihrer Lan-Anschl"usse. Kennt eine
        Bridge bereits einen g"unstigen Weg in Richtung Wurzel, sendet sie
        dessen Kosten auf alle angeschlossenen LANs. Ein Bridge, die diese
        Information empf"angt, w"ahlt jetzt den Nachbarn aus, "uber den der
        optimale Pfad zur Wurzel aus. (d.h. eigene Kosten bis zum Nachbarn +
        dessen Kosten bis zur Wurzel sind minimal). Rahmen werden von Bridges
        jetzt nur noch entlang des Baumes weitergeleitet, ansonsten ist alles
        wie vorher.
        
        Ein LAN wird seinen Verkehr immer "uber diejenige Bridge versenden, die
        die beste Anbindung zur Wurzel hat (designated Bridge).
      \subsubsection*{Source Routing Bridges}
        Diese Art von Bridges verlangt, da"s die sendende Station den kompletten
        Pfad bis zum Ziel eintr"agt. Dies macht die Arbeit der eigentlichen
        Bridge zwar einfacher, ist aber nicht transparent. Au"serdem m"ussen
        alle Stationen die Netztopologie kennen, was mit einem hohen Aufwand
        (z.B. quadratisch viele Nachrichten) verbunden ist. Diese Technik wurde
        bei Bridges f"ur 802.5 verwendet.
    \subsection*{Spezielle MAC-Systeme}
      \subsubsection*{802.3 (Ethernet)}
        Ethernet hat den Vorteil simpel (und billig) zu sein, eine Eigenschaft
        die sich durchsetzt. Ethernet arbeitet mit einem CSMA/CD-Verfahren mit
        exponentiellem Backoff. Es gibt verschiedene Verkabelungsvarianten
        (Thick, Thin, Twisted Pair, Glasfaser) und inzwischen auch verschiedene
        Geschwindigkeiten. In jedem Fall kann jedoch jede angeschlossene Station
        alle anderen h"oren, und die maximale L"ange eines Segmentes betr"agt
        2500 Meter (wichtig f"ur die Kollisionserkennung).
        
        Ein Ethernet-Rahmen besteht aus einer 7-byte-Pr"aambel (01010...), die
        zur Synchronisation dient. Dann folgt ein Startcode (10101011) als
        Anfangsmarkierung. Die Quell- und Zieladresse haben jeweils 6 Byte
        (m"oglich w"aren auch 2), dann folgen 2 Byte die die L"ange der Daten
        angeben (max. 1500 Byte). Falls der Rahmen weniger als 64 Byte lang
        w"are, wird ein Padding-Feld eingef"ugt um ihn auf diese Mindestgr"o"se
        zu verl"angern (die Mindestgr"o"se ist erforderlich, um bei einer L"ange
        von 2500m noch Kollisionen zu erkennen. Der Rahmen wird durch eine
        4-byte-Pr"ufsumme (CRC) gesch"utzt. Ethernet-Adressen k"onnen lokal oder
        global sein, die Adresse 111.... dient als Broadcastadresse.
        
        Der bin"are Backoff verdoppelt die Gr"o"se des Konkurrenzfensters bis zu
        einem Maximalwert von 1023 Schlitzen, dann (nach der 10 Kollision) wird
        die Gr"o"se eingefroren. Nach 16 Kollisionen gibt der Algorithmus auf.
        
        Aufgrund seiner Einfachheit hat Ethernet auch einige Nachteile: Die
        Zustellung eines Rahmens in einer gewissen Zeit kann nicht garantiert
        werden und es werden keine Priorit"aten unterst"utzt. Ethernet eignet
        sich also nicht f"ur Echtzeitanwendungen. Au"serdem ist die m"ogliche
        Auslastung unter hoher Last nicht gerade optimal.
        
        \textbf{Fast Ethernet} ist im Prinzip das gute alte Ethernet, nur da"s
        der Takt der Manchester-Codierung verzehnfacht wurde und so statt 10
        Mbit/s jetzt 100 Mbit/s "ubertragen werden k"onnen. Fast Ethernet
        unterst"utzt nur noch die Verkabelung "uber Hubs, mit Kabel der
        Kategorien 3 und 5 oder Glasfaser.(Kat 3 arbeitet mit vier verdrillten
        Kabelpaaren (zum Hub, vom Hub, und zwei umschaltbare).
        
        \textbf{Switches} k"onnen Ethernet nochmals verbessern, indem der Switch
        Pakete an eine bestimmte Station nur an den Ausgang dieser Station
        kopiert. Andere Stationen werden also von dieser Kommunikation nicht
        bel"astigt.
      \subsubsection*{802.4 (Token Bus)}
        Token Bus war der Versuch die Probleme des Ethernet zu l"osen und ein
        System zu schaffen, da"s sich besonders f"ur Fertigungsstrassen eignet.
        Token Ring ist eine unn"otig komplexe Kopfgeburt, die wohlverdient
        untergegangen ist.
        
        Das Grundprinzip ist noch relativ elegant: Obwohl die Stationen
        physikalisch an einem Bus h"angen, bilden sie einen logischen Ring. Eine
        Station darf nur senden, wenn sie das Token besitzt und sendet es,
        nachdem sie fertig ist an die logisch n"achste Station weiter. Dadurch
        kommt jede Station garantiert regelm"a"sig an die Reihe. Allerdings
        besteht jede Station aus sechs Unterstationen, die jeweils eine der
        sechs Priorit"atsklassen bedienen. Der h"ochsten Priorit"at wird ein
        gewisser Anteil der Bandbreite garantiert.
        
        Die Rahmen von Token-Ring unterscheiden sich etwas von den
        Ethernet-Rahmen. Die Pr"aambel ist nur ein Byte lang, und jeder Rahmen
        hat ein Start- und Endesignal das analog (!) codiert ist, deshalb mu"s
        keine Rahmenl"ange "ubertragen werden.
        
        Die Probleme von Token Ring fangen richtig an, wenn zus"atzliche
        Stationen eingef"ugt werden m"ussen. Jede Station kann bietet, wenn sie
        das Toke hat, anderen Stationen (und zwar nur solchen, die nach ihr in
        den Ring einf"ugt werden k"onnen) an, ihr Nachfolger zu werden. Wenn
        sich mehrere Stationen melden, treten Kollisionen auf, und die
        urspr"ungliche Station versucht dies zu beheben. Die genaue Zeit, um in
        den Ring einzutreten, l"a"st sich nicht vorhersagen.
        
        Werden beide Nachfolgestationen einer Station abgeschaltet (d.h. sind
        pl"otzlich nicht erreichbar, mu"s der Ring neu initialisiert werden.
        
        Token Ring untest"utzt bis zum 10 MBit/s.
      \subsubsection*{802.5 (Token Ring)}
        Der Grundgedanke von Token Ring ist derselbe wie beim Token Bus: Nur die
        Station, die gerade das Token besitzt, darf senden. Der auff"alligste
        Unterschied besteht darin, da"s ein Token Ring auch physikalisch als
        Ring organisiert ist, d.h. eine Station kommuniziert nur mit ihren
        unmittelbaren Nachbarn direkt. Jede Station hat eine Ein- und eine
        Ausgangsschnittstelle. Normalerweise sind die Stationen
        miteinander "uber ein \textit{Wire Center} verbunden, das eine
        ausgefallenen oder unterbrochene Station "uberbr"ucken kann.
        
        Eine Station kopiert im Normalfall (Lesebetrieb) alle Daten von ihrer
        Eingangschnittstelle mit einer 1-Bit Verz"ogerung auf ihre
        Ausgangschnittstelle. Falls die Station das Schreibrecht hat, werden
        Ein- und Ausgang entkoppelt, die Station sendet auf ihrem Ausgang Daten
        und leitet die Daten, die sie auf dem Eingang erh"alt, nicht weiter.
        
        Im Ruhezustand kreist auf dem Ring ein 3 Byte langes Token, das von den
        Stationen weitergegeben wird. Der Ring mu"s mindestens so lang sein,
        da"s das komplette Token auf ihm Platz findet, ist dies nicht der Fall,
        wird eine zus"atzliche Verz"ogerung eingef"ugt. Sobald eine Station
        senden will, wartet sie auf das Token, "andert ein Bit um es zu einer
        normalen Startfolge zu machen und sendet dann den Rest des Rahmens. Die
        Station darf jetzt bis zu einer maximalen Tokenhaltezeit Rahmen senden,
        sobald sie fertig ist, erzeugt sie ein neues Token und geht zur"uck in
        den Lesebetrieb.
        
        Ein Token-Ring-Rahmen besteht aus eine Start- und einem Endebegrenzer
        von einem Byte die ung"ultige Manchester-Folgen enthalten um sie von
        Daten zu unterscheiden. Nach dem Endebegrenzer folgt noch ein
        Rahmenstatusbyte, in dem die Empfangstation Best"atigungen bzw. den
        "Ubertragungsstatus eintragen kann. Der Kopf des Rahmens besteht au"ser
        aus dem Startbegrenzer aus einem Zugriffsteuerungsbyte, da"s Bits f"ur
        Reservierung, Priorit"aten und "Uberwachungszwecke enth"alt. Das
        Rahmensteurungsbyte codiert verschiedene Arten von "Uberwachungsrahmen.
        Der Rest des Token-Ring Rahmens besteht aus Start- und Zieladresse, der
        Checksumme und den Daten, die hier beliebig lang sein d"urfen.
        
        Token Ring unterst"utzt drei \textbf{Priorit"aten}, 
        eine Station darf nur senden
        wenn sie ein Token gleich oder kleiner der anstehenden Priorit"at
        erh"alt. Eine Station kann ein Reservierungsbit setzen, um ein Token
        einer gewissen Priorit"at anzufordern, eine Reservierung mit hoher
        Priorit"at hat dabei Vorrang. Es gibt einige Mechanismen, die erreichen
        sollen, da"s die Priorit"at nach der "Ubertragung auch wieder
        heruntergesetzt wird, allerdings kann es vorkommen da"s eine Station mit
        niedriger Priorit"at verhungert.
        
        Eine \textbf{Monitorstation} "ubernimmt in Token Ring die
        Administrationsaufgaben, f"ugt Verz"ogerungen ein falls notwendig,
        entfernt kreisende Rahmen ohne Heimat und sonstigen M"ull vom Ring und
        "ahnliches. Jede Station kann potentiell der Monitor des Rings sein,
        beim Ausfall des Monitors "ubernimmt eine andere Station diesen Job. Es
        kann nat"urlich vorkommen, da"s der Monitor eine Fehlfunktion hat und
        damit das Netz zum Erliegen bringt (Anh"anger von Token Bus glauben
        daher, ihr System w"are zuverl"assiger...).
        
        Token Ring unterst"utzt bis zum 8(16) MBit/s.
      \subsubsection*{802.11 (Wireless LAN)}
        802.11 ist ein neues Miglied in der 802.x-Familie zum Aufbau von
        drahtlosen Netzwerken. Naturgem"a"s ist es etwas komplizierter als die
        drahtgebundenen Protokolle: 802.11 unterst"utzt verschiedene
        "Ubertragungsarten, verschiedene MAC-Zugriffsverfahren, Stromsparmodi
        und Roaming.
        
        Die \textbf{"Ubertragungsarten}, die unsterst"utzt werden sind eine
        Funk"ubertragung mit Frequenzsprungverfahren, eine Funk"ubertragung mit
        DSSS (Direct Sequence Spread Spectrum) und Infrarot"ubertragung. Es
        wird eine "Ubertragungsgeschwindigkeit von 1 MBit/s unterst"utzt,
        optional 2 MBit/s, moderne Varianten unterst"utzen auch 11 MBit/s. Die
        \textbf{Infrarot"ubertragung} hat dabei nur eine Reichweite von ca. 10
        Metern, und setzt keine direkte Sichtverbindung voraus. Die
        Funkbasierten Verfahren arbeiten im lizenfreien Bereich um 2,4 GHz und
        haben eine Reichweite im Bereich um 50m (in Geb"auden). Die
        "Ubertragungstechniken haben jeweils eigene Paketformate f"ur die
        physikalische Schicht. Alle drei Verfahren bieten ein CCA (Clear Channel
        Assement) Signal an, das angibt ob das Medium frei ist. Bei beiden
        Funkverfahren werden die Daten mit einem Polynom zerw"urfelt
        (scrambling), um Gleichstromanteile zu vermeiden und das Signal zu
        gl"atten.
        
        Ein Verfahren zur Funk"ubertragung ist ein 
        \textbf{Frequenzsprungverfahren}. Die Daten werden mit einer
        Gausschen Frequenzmodulation codiert, es werden 2 (1 MBit/s) bzw. 4 (2
        MBit/s) Frequenzen benutzt. Die Station wechselt die Kan"ale mit einer
        pseudozuf"alligen Sprungsequenz, so da"s mehrere Netzwerke im gleichen
        Raum betrieben werden k"onnen. Es stehen 79 Kan"ale (Japan: 23) zur
        Verf"ugung. Das Paket dieser Schicht besteht aus einer Pr"aambel (80 bit
        Synchronisation und 2 Startbyte), einer L"angenangabe f"ur die Folgenden
        Nutzdaten (12 Bit), einem 4-bit Steuerungscode der die Datenrate der
        Nutzdaten angibt (der Header wird immer bei 1 MBit/s "ubertragen) und
        einer 4 byte langen Checksumme. Danach folgen die Nutzdaten.
        
        Das am weitesten verbreitete Verfahren ist allerdings \textbf{DSSS}.
        Hier werden die Daten per Phasenmodulation "ubertragen (Modulation je
        nach "Ubertragungsrate), und das Signal wird mit einem 11 Bit langen
        Barker-Code gespreitzt.  Dieses Verfahren ist stabiler, erforder
        allerdings auch leistungsf"ahigere Sender und Empf"anger. Eine neue
        Variante des DSSS-Verfahren kann bereits Datenraten von 11 MBit/s
        "ubertragen. Der Datenrahmen der DSSS-Schicht unterscheidet sich nur
        unwesentlich von dem beim Frequenzsprungverfahren. Es gibt wieder eine
        Pr"aambel (diesmal mit 128 Bit Synchronisationsmuster) und Felder f"ur
        die L"ange der Daten, die "Ubertragungsgeschwindigkeit und eine
        Pr"ufsumme. Zus"atzlich ist ein Feld f"ur sp"atere Verwendung
        hinzugekommen, da"s f"ur weitere Dienste verwendet werden soll. Die
        Felder haben teilweise etwas andere L"angen als beim
        Frequenzsprungverfahren.
        
        Die \textbf{MAC-Schicht} in 802.11 unterst"utzt wiederum verschiedene
        Verfahren: Ein CSMA/CA-Verfahren f"ur ad-hoc Netzwerke, ein optionales
        Verfahren mit RTS/CTS und ein kollisionsfreies Polling-Verfahren mit
        einer Masterstation. Die Verfahren verwenden drei verschiedene
        Wartezeiten, eine sehr kurze (SIFS) zum Kollisionsfreien Versenden von
        Best"atigungen u."a., eine mittler f"ur den privilegierten Zugriffe 
        (PIFS) der Masterstation und eine lange Wartezeit (DIFS) 
        f"ur konkurrierenden Zugriff. 
        
        Beim \textbf{CSMA/CA-Verfahren} pr"uft eine Station zuerst (mit dem
        CCA-Signal), ob das Medium belegt ist. Ist es f"ur mindestens die Zeit
        von DIFS frei, kann sie sofort zu senden beginnen. Ist das Medium
        belegt, m"ussen alle Stationen die senden wollen zun"achst eine Zeit von
        DIFS abwarten. Ist das Medium dann immer noch frei, treten sie in eine
        Konkurrenzperiode ein: Jede Station w"ahlt eine zuf"allige Zahl, und
        wartet nach DIFS noch entsprechend viele Zeitschlitze ab. Beginnt vorher
        eine andere Station zu senden, bricht sie ihren Versuch ab. Beginnen
        zwei Station im selben Zeitschlitz, kommt es zu einer Kollision.
        Stationen, die ihren Versuch abgebrochen haben, m"ussen nur noch die
        \glqq Restzeit\grqq\ abwarten, bis sie es wieder versuchen k"onnen.
        Falls Kollisionen auftreten, wird wieder ein exponentielles
        Backoff-Verfahren verwendet: Die Anzahl der Zeitschlitze ist minimal 7,
        nach jeder Kollision wird sie bis zu einem Maximalwert von 256
        verdoppelt. Ein Paket wird vom Empf"anger sofort best"atigt, die
        Wartezeit bis zum Senden der Best"atigung betr"agt nur SIFS, so da"s die
        Best"atigung in jedem Fall Kollisionsfrei versendete wird. 
        
        CSMA/CD kann mit einem
        \textbf{RTS/CTS}-Mechanismus kombiniert werden. Jede Station mu"s diesen
        Mechanismus unterst"utzen, aber die Anwendung ist optional: RTS/CTS
        erzeugt eine Menge zus"atzlicher Nachrichten, und kann oft weggelassen
        werden. Wenn eine Station den Mechanismus anwenden will, sendet sie (im
        normalen Konkurrenzverfahren einen RTS-Rahmen, der die Adresse des
        Empf"angers und die L"ange der Daten"ubertragung enth"alt. Alle
        Stationen, die diesen Rahmen h"oren passen ihren NAV (Net Allocation
        Vector) an, und werden das Medium f"ur die vorgegebene Zeit nicht
        belegen. Der Empf"anger sendet dann (nach SIFS) ein CTS-Signal, alle
        Stationen die dieses h"oren passen ihren NAV ebenfalls (neu) an. Danach
        erfolg die Sendung, die von keiner anderen Station unterbrochen wird.
        802.11 unterst"utzt auch ein \textbf{Fragmentierungsverfahren} (damit
        nicht zu gro"se Bl"ocke "ubertragen werden. In diesem Fall enth"alt die
        Best"atigung eines Fragmentes gleichzeitig die Reservierung des Netzes
        f"ur das n"achste Paket.
        
        Das \textbf{Polling-Verfahren} teilt die Zeit in sogenannte Superrahmen
        ein, die aus einer Polling-Phase und einer Konkurrenzphase bestehen
        (diese funktioniert mit CSMA/CD). Da die Konkurrenzphase das Polling
        verz"ogern kann, kann sie auch ganz weggelassen werden. Beim Polling
        fragt die Masterstation (nach PIFS, also bevor jemand anderes drankommt)
        die einzelnen Stationen ab, und diese k"onnen auf die Abfrage mit ihren
        Daten (nach SIFS) antworten. Darauf fragt die Masterstation (nach SIFS)
        die n"achste Station ab. Die Masterstation zeigt den Beginn der
        Konkurrenz mit einem bestimmten Signal (Contention Free End) an.
        
        Der \textbf{MAC-Rahmen} von 802.11 enth"alt Kontrollinformationen, eine
        Sequenznummer, Daten "uber die Dauer der "Ubertragung (f"ur die
        RTS/CTS-Reservierung), Daten (bis 2312 bit) und 4 Adressen. Diese dienen
        zur Unterst"utzung des Roaming. Wenn die Stationen untereinander
        kommunizieren, enthalten zwei der Adressfelder die benutzten Adressen,
        das dritte die Adresse des Netzwerkes (BSSID Basic Service Set
        Identifier). Werden Daten "uber einen Zugangspunkt versandt, ist sowohl
        die physikalische Adresse des Zugangspunktes als auch die logische
        Adresse des Partners enthalten.
        
        Wichtig f"ur die Stromsparmodi ist eine genaue
        \textbf{Zeitsynchronisation}. Zu diesem Zweck sendet die Masterstation
        (falls vorhanden) in regelm"assigen Abst"anden ein Timestamp-Paket
        (Beacon) aus.
        Falls es sich um ein Ad-Hoc-Netzwerk handelt, wird jede Station zum
        vorgesehenen Zeitpunkt versuchen ein Beacon zu senden, auf das
        sich die anderen synchronisieren. Die Pakete werden immer m"oglichst
        nah am vorgesehenen Zeitpunkt gesendet. Verschiebt sich ein
        Timestamp-Paket, verschieben sich die nachfolgenden nicht.
        
        Um \textbf{Strom zu sparen} kann sich eine 802.11-Station vor"ubergehend
        abschalten. Andere Stationen m"ussen dann die Daten speichern, bis sie
        wieder erwacht. Gibt es eine Masterstation, "ubernimmt diese das
        Zwischenspeichern der Daten. Die anderen wachen regelm"a"sig auf, um das
        Beacon zu h"oren. Zusammen mit dem Beacon wird auch eine Traffic
        Idenfication Map gesendet, die angibt f"ur welche Stationen Daten
        vorliegen. Diese bleiben dann wach. In einem Ad-Hoc-Netzwerk "ubertr"agt
        jede Station die senden will eine (A)TIM, die entsprechende Station
        bleibt dann wach. Dies skaliert schlecht, da bei einem vollen Netz sehr
        viele Tabellen "ubertragen werden.
      \subsubsection*{802.6 DQDB}
        Distributed Queue Dual Bus: Ein Protokoll f"ur MANs (ungef"ahr die
        Gr"osse einer Stadt). Die Besonderheit ist, da"s es zwei unidirektionale
        Leitungen gibt, auf denen die Pakete weitergereicht werden. Die Station
        sind als dezentrale FIFO-Warteschlange reserviert: Eine Station kann die
        anderen zwingen, ihr einen Rahmenschlitz zur reservieren indem sie in
        der Gegenrichtung ein Reservierungsbit setzt.
      \subsubsection*{802.2 LLC f"ur 802.x}
        802.x sieht keine Best"atigungen oder zuverl"assige Verbindungen vor.
        802.2 definiert daher ein einheitliches LLC (Logical Link Control)
        Protokoll, das einen einheitlichen LLC-Rahmen f"ur alle 802.x-Netzwerke
        vorsieht. Dies ist eine Teilschicht oberhalb der MAC-Schicht. LLC bietet
        best"atigten und unbest"atigten Datagrammdienst und einen
        verbindungsorientierten Dienst.
      \subsubsection*{FDDI}
        Fiber Distributed Data Interface - ist im Prinzip ein Token Ring auf
        Glasfaserbasis mit einer Geschwindigkeit von 100 MBit/s. Es benutzt
        allerdings eine Priorit"atsregelung "ahnlich von Token Bus: Solange das
        Token voraus ist, d"urfen alle senden, ansonsten nur die hohen
        Priorit"aten. FDDI braucht eine lange Pr"aambel zur Synchronisation, da
        es nicht Manchester-codiert.
      \subsubsection*{HIPPI}
        High Performance Parallel Interface - Eine
        Supercomputer-Verbindungschnittstelle mit 800 MBit/s. Das ganze
        funktionierte schon in den 80er Jahren mit einem Kabel mit 50
        verdrillten Kabelpaaren, davon eines in jede Richtung. Will man die
        doppelte Bandbreite, nimmt man vier Kabel. Das ganze funktioniert
        immerhin auf 25m und wurde so eine Art Industriestandard bei
        Supercomputern.
      \subsubsection*{Fiber Channel}
        Der designierte Nachfolger von HIPPI auf Glasfaserbasis. Kommt komplett
        mit einem Kann-Alles-Gewinnt-Immer-Protokoll und bis zu 800 Mbit/s.
  \section{Vermittlungsschicht}
    Die Vermittlungsschicht (Network Layer) hat die Aufgabe, Pakete zwischen
    zwei Kommunikationspartnern zu "ubertragen, und zwar auch "uber
    zwischengeschaltete Router o."a. Dies ist die erste Ende-zu-Ende Schicht.
    Auch wenn IP (die Internet-Vermittlungsschicht mit Paketen arbeitet, kann
    eine Vermittlungsschicht (wie bei ATM) auch zuverl"assige und/oder
    verbindungsorientierte Dienste anbiete.
    \subsection*{Routing}
      Dies ist eine der Hauptaufgaben der Vermittlungsschicht: Eine Route vom
      Sender bis zum Empf"anger zu finden. Allgemein kann man das Netz als
      Graphen auffassen, bei dem die Knoten (=Router) "uber die Kanten
      Informationen austauschen. Jede Kante hat ein Gewicht, das die Entfernung
      zwischen den Routern sein kann oder die Antwortzeit oder sonst etwas
      sinnvolles. Kennt man den gesamten Graphen, ist es m"oglich eine optimale
      Route von einem Knoten zu berechnen (z.B. mit Dykstra's Algorithmus).
      Allerdings wei"s nicht immer jeder Router "uber alles bescheid, und die
      Toplogie des Netzes "andert sich oft. (Es ist auch m"oglich, statt einer
      einzelnen Route die Leistung des ganzen Netzes zu optimieren, wenn man den
      "ublichen Verkehrsflu"s kenn: \textbf{Flu"sbasiertes Routing})
      
      Also mu"s man ein Verfahren finden, mit dem jeder Router erf"ahrt, was zu
      tun ist wenn ein Paket empfangen wird.
      \subsubsection*{Flooding}
        Dieser Mechanismus hat den Vorteil, da"s er immer funktioniert, ist aber
        nicht sehr effizient: Wenn ein Router ein Paket erh"alt, wird es auf
        allen Leitungen wieder ausgegeben, au"ser der, auf der es gekommen ist.
        Die Pakete m"ussen mindestens eine vorgegebene Lebensdauer haben (z.B.
        maximalen Hop-Count), damit keine endlos kreisenden Pakete entstehen.
      \subsubsection*{Backward Learning}
        Flooding leidet darunter, da"s es einen Haufen unn"otiger Pakete
        erzeugt. Beim Backward Learning (z.B. bei Bridges eingesetzt), verwendet
        der Router Flooding, falls er den Weg zu einem Ziel nicht kennt.
        Au"serdem wird die Ursprungsadresse des Paketes untersucht: Die Leitung,
        auf der es ankam ist offensichtlich der richtige Weg zum entsprechenden
        Subnetz. Auf diese Weise baut sich der Router (oder die Bridge) eine
        Tabelle auf, die Eintr"age haben nur eine gewisse Lebensdauer damit auch
        auf "Anderungen der Topologie reagiert werden kann.
        
        Backward Learning funktioniert nur, wenn es keine Schleifen in der
        Topologie gibt.
      \subsubsection*{Hot Potato}
        Ein besonders lustiges Verfahren: Wir versuchen das Paket m"oglichst
        schnell wieder los zu werden und stellen es in den Ausgang mit der
        k"urzesten Warteschlange.
      \subsubsection*{Statisches Routing}
        Die Routingtabellen werden an den Routern von Hand eingestellt.
        Funktioniert prima, solange die Hierarchie sich nicht "andert.
      \subsubsection*{Zentralisiertes- und Delta Routing}
        Man kann eine zentrales Routing Control Center einrichten, da"s alle
        Routingentscheidungen trifft. Es mu"s einen Gesamt"uberblick "uber das
        Netz haben (etwa indem es Statusmeldungen von allen Stationen
        einsammelt), berechnet f"ur jeden Router die Routingtabelle und schickt
        sie ihm. Das Routing Control Center darf nie ausfallen und mu"s
        ausreichend leistungsf"ahig sein.
        
        Beim Delta Routing bestimmt das Kontrollzentrum "aquivalente Routen in
        der Routingtabelle (d.h. Routen, die sich nur um maximal ein Delta
        unterscheiden). Der Router darf dann selbst entscheiden, welchen der
        "aquivalenten Ausg"ange er benutzen will.
      \subsubsection*{Distance Vektor Routing}
        Ein dynamisches Verfahren, bei dem jeder Router nur "uber seine
        Nachbarschaft bescheid wei"s. Jeder Router wei"s, zu welchen Kosten er
        einen anderen Router (am Anfang seine Nachbarn) erreichen kann. Diese
        Information teil er seinen Nachbarn mit. Merkt jetzt z.B. Router A, da"s
        sein Nachbar B den Router C mit 1 Hops erreichen kann, wei"s A da"s er C
        mit 2 Hops erreichen kann (indem er das Paket an B sendet). Diese
        Information teilt er wieder seinen Nachbarn mit.
        
        F"allt allerdings C aus, so kann B ihn nicht mehr direkt erreichen. B
        wei"s aber, das A behauptet C in 2 Hops erreichen zu k"onnen. Also
        glaubt B, es k"onne C "uber A in 3 Hops erreichen. Wenn A dies h"ort,
        glaubt es C "uber B in 4 Hops erreichen zu k"onnen, und so weiter.
        (\textbf{Count-To-Infinity-Problem}). Es gibt einige Optimierungen, die
        dieses Problem mildern sollen (z.B. Split Horizon: Der Router schickt
        die Entfernungsinformationen nicht auf die Leitung, mit der er den
        entsprechenden Partner erreicht), diese Hacks funktionieren aber alle
        nicht zuverl"assig. Das Border Gateway Protokoll vermeidet jedoch das
        Problem, indem komplette Pfade "ubertragen werden.
      \subsubsection*{Destination Sequence Distance Vektor Routing}
        Eine Variante des Distance Vektor Routing, die f"ur mobile ad-hoc-Netze
        gedacht ist. Das Problem dort ist, das die Topologie sich schnell
        "andert, und sich das Routing schnell anpassen mu"s. Insbesondere
        count-to-infinity ist hier absolut t"odlich.
        
        Diese Variante verwendet zus"atzlich eine Sequenznummer f"ur jede
        gesendete Routinginformation (um Schleifenbildung zu vermeiden), 
        und merkt sich deren Alter. "Anderungen werden erst nach einer kurzen
        Bedenkzeit verwendet, um kurzfristige Schwankungen auszugleichen.
      \subsubsection*{Dynamic Source Routing}
        Ein weiteres Verfahren f"ur mobile ad-hoc-Netze. Das eigentliche Routing
        entspricht dem bei den Bridges erkl"arten Source Routing: Jeder Sender
        schickt den kompletten Pfad bis zum Ziel in seiner Nachricht. Allerdings
        wird hier nicht die komplette Netztopologie ausgetauscht, sondern ein
        Knoten versucht nur einen Pfad zu finden, wenn er etwas zu senden hat.
        
        Kennt der Sender den ben"otigten Pfad noch nicht, so wird die Nachricht
        geflutet, und jeder Router h"angt seine Adresse an das Paket an. Der
        Empf"anger erh"alt dann Pakete mit kompletten Pfadangaben, kann den
        g"unstigsten aussuchen und mit der Best"atigung an der Sender
        zur"uckschicken. 
        
        Dieses Verfahren hat den Vorteil, nicht st"andig Routinginformationen
        auszutauschen. Es wird aber problematisch in hochdynamischen Umgebungen
        in denen sich die Pfade immer "andern, und wenn die Verbindungen
        zwischen zwei Knoten nicht symmetrisch sind.
      \subsubsection*{Link State Routing}
        Hier sammelt jeder Router Informationen "uber die Verbindungen zu seinen
        Nachbarn und stellt diese in einem Link-State-Paket zusammen. Diese
        Pakete werden dann (z.B. per Flooding) an alle anderen Router verteilt.
        Die Link-State-Pakete haben (zur Vermeidung von Zweideutigkeiten) eine
        Sequenznummer (um die richtige Reihenfolge festzustellen) und eine
        maximale Lebensdauer (die heruntergez"ahlt wird, damit z.B. nicht ewig
        Pakete mit niedriger Sequenznummer nach einem Neustart verworfen
        werden).
        
        Hat ein Router die Link-State-Pakete von allen anderen Routern erhalten,
        kann (z.B. mit Dijkstras Algorithmus) die optimalen Routen berechnen.
      \subsubsection*{Hierachisches Routing}
        Um die Routingentscheidungen nich endlos komplex werden zu lassen, kann
        man Netze als Hierachien auffassen. Innerhalb eines Teilnetzes wird ein
        beliebiges Routingverfahren verwendet, alle Pakete die das Teilnetz
        verlassen gehen an einen besonders designierten Router. Dieser ist Teil
        der n"achsth"oheren Hierarchiestufe, in der das Routing ohne Betrachtung
        der tieferen Stufen erfolgt.
      \subsubsection*{Broadcast/Multicast}
        Sollen mehrere Empf"anger angesprochen werden, kann man nat"urlich
        einfach das Netz fluten. Eine Alternative ist, da"s jeder Router das
        Broadcast/Multicast-Paket auf allen Leitungen verschickt, die eine
        korrekte Route zum einem der Ziele darstellen
        (Multidestination-Routing). Eine weitere Methode ist, einen
        \textit{Spanning Tree} "uber alle Knoten aufzubauen, und die Pakete an
        dessen Kanten weiterzuleiten - dazu mu"s aber jeder Knoten die komplette
        Netztopologie kennen. Beim Multicast existiert zus"atzlich ein Spanning
        Tree f"ur jedes Mitlied der Gruppe (jeder ist beim Senden die Quelle des
        Spanning Tree) was zu Speicherproblemen f"uhren kann. (Alternativ kann
        auch nur ein Baum existieren, und die Daten werden zuerst an die Wurzel
        gesendet. (Beim Multicast existiert zus"atzlich das Problem, den Baum
        auf alle Mitglieder der Gruppe zu beschneiden.)
        
        Ein Verfahren, um auch ohne komplette Kenntnis der Topologie einen Baum
        zu erhalten ist das \textbf{Reverse Path Forwarding}. Hier nimmt ein
        Router an, da"s ein Multicast-Paket das auf der "ublichen Route von
        seinem Sender ankommt, das \glqq richtige \grqq\ ist. Alle Pakete, die
        nicht "uber die Leitung kommen, "uber die normalerweise Pakete an den
        Sender gehen, werden als Duplikate verworfen, das \glqq richtige \glqq\
        Paket wird weitergeflutet. Mit diesem Ansatz wird das Verhalten eines
        spannenden Baumes nachgebildet. Beim Multicast-Routing kann ein Router,
        der keine Hosts der entsprechenden Gruppe besitzt verlangen, da"s er
        keine Pakete mehr erh"alt (PRUNE). Ein Router, der von allen seinen
        Nachbarn PRUNE-Meldungen bekommt, kann sich ebenfalls ausklinken (dieser
        Mechanismus entspricht dem Beschneiden des Baumes).
    \subsection*{"Uberlastungs"uberwachung}
      Die Steuerung von "Uberlastungen unterscheidet sich etwas von der
      Flu"skontrolle: Hier geht es darum, da"s das Netz als solches die Last
      nicht mehr verkraften kann (z.B. ein Router ist "uberlastet) - es geht
      also nicht um die beiden Endsystem, sondern um das, was dazwischenliegt.
      Auf der Vermittlungsschicht geht es haupts"achlich darum, da"s Router die
      Datenmenge, die sie erhalten, nicht mehr richtig verarbeiten k"onnen.
      
      Bei der Umschiffung von "Uberlastungen gibt es zwei M"oglichkeiten:
      Entweder das System so zu konstruieren, da"s keine "Uberlastungen
      entstehen (offene Schleife) oder zu versuchen, bei "Uberlastungen
      Gegenma"snahmen zu treffen (geschlossene Schleife). Manche Methoden
      funktionieren besser (oder nur) mit virtuellen Verbindungen, manche
      eignen sich auch Datagrammdienste.
      \subsubsection*{Traffic Shaping}
        Hier sollen die Station gezwungen werden, ihre Daten in einer
        einergerma"sen vorhersehbaren Rate zu senden, was die Vermeidung von
        "Uberlastungen ungemein vereinfacht. ATM wendet z.B. Traffic-Shaping an
        um eine maximale Auslastung der virtuellen Verbindungen zu garantieren.
        Traffic Shaping geht davon aus, da"s die Transportschicht einfach Daten
        sendet, eventuell mit pl"otzlichen Spitzen, oder viel zu viele.
        
        Um das wieder in den Griff zu bekommen, die Datenrate zu beschr"anken
        und etwas gleichm"a"siger zu machen, kann man den
        \textbf{Leaky-Bucket}-Algorithmus verwenden. Er modelliert einen Eimer,
        in den (schwallweise?) Wasser gesch"uttet wird, das durch ein Loch im
        Boden wieder heraustropft. L"auft der Eimer "uber, geht Wasser verloren.
        Praktisch l"a"s also der Router immer nur eine gewisse Anzahl Pakete pro
        Zeittakt passieren und sorgt damit f"ur eine gleichm"a"sige Datenrate. 
        Kommen mehr Pakete an, werden sie gepuffert, l"auft der Puffer "uber,
        werden sie verworfen. 
        
        Will man kurz\-zeitige Spitzen zulassen, kann man den
        \textbf{Token-Bucket} Algorithmus verwenden. Hier nimmt der Eimer Tokens
        auf, die in einer gleichm"a"sigen Rate erzeugt werden. Jedes Paket, das
        verschickt wird, verbraucht ein Token. Wenn eine Sendung beginnt, kann
        sie solange mit voller Geschwindigkeit senden, bis die Token im Eimer
        verbraucht sind. Danach kann nur noch mit der Geschwindigkeit gesendet
        werden, wie neue Token erzeugt werden. Beide Mechanismen lassen sich
        statt mit Paketen auch byteweise durchf"uhren.
        
        Damit klar ist wie viele Resourcen gebraucht werden, k"onnen die
        Partner (die Stationen und das Netz) eine Flu"sspezifikation
        vereinbaren. Hier werden die maximalen Datenraten, zul"assige
        Verz"ogerungen, die Gr"o"se der Buckets und "ahnliches festgelegt.
      \subsubsection*{Choke-Pakete}
        Wenn ein Netz "uberlastet ist, hilft es irgenwann nur noch, wenn die
        Sender ihre Last zur"uckfahren. Eine M"oglichkeit, die Sender von der
        "Uberlast zu informieren, sind sogenannte \textit{Choke-Pakete}. Dies
        sind Warnungen die (eventuell huckepack) von der "uberlasteten Stelle an
        den Sender geschickt werden, damit dieser seine Sendung reduziert.
        Erh"alt der Sender weitere Choke-Pakete, drosselt er die Leistung
        weiter, kommen keine mehr legt er langsam wieder zu. Choke-Pakete habe
        allerdings das Problem bei einer "Uberlassituation noch weitere Pakete
        zu erzeugen, au"serdem kann die Reaktionszeit auf langen Leitungen recht
        lang sein. Letzteres Problem kann behoben werden, wenn jeder Router, der
        ein Choke-Paket erh"alt die Leistung sofort drosselt und die restlichen
        Pakete puffert (\textbf{Hop-by-Hop-Choke}).
      \subsubsection*{Weighted Fair Queuing}
        Bei den Choke-Funktionen (nicht nur da) kann es passieren, da"s sich ein
        Sender nicht an die Vereinbarungen h"alt und dadurch unfairerweise
        bevorzugt wird (weil er als einziger seine Last nicht drosselt). Um ihn
        daran zu hindern kann jeder Router f"ur jeden Eingang eine eigene
        Warteschlange haben, diese werden dann reihum bedient. Ein \glqq
        unfairer\grqq\ Router bekommt so nicht mehr Bandbreite als einer, der
        sich korrekt verh"alt. Soll eine Leitung bevorzugt werden, so darf sie
        mehr als ein Paket pro Zeiteinheit versenden. (Das ganze funktioniert
        wieder auch byteweise statt paketweise).
      \subsubsection*{Load Shedding}
        Gibt es keine andere M"oglichkeit, mu"s der Router irgendwann anfangen,
        Daten zu verwerfen. Damit sollte er auch noch einigerma"sen fr"uhzeitig
        anfangen, bevor die Situation v"ollig verfahren ist. Die Frage ist jetzt
        nur noch: Welche Daten k"onnen weg? Sind die Daten z.B. ein
        Multimedia-Strom, k"onnen eher alte Pakete verworfen werden (Milch). 
        In anderen F"allen sind neue Pakete weniger wichtig (Wein). In anderen
        F"allen k"onnte der Router sinnvollere Entscheidungen treffen, wenn er
        den Inhalt der Daten kennt. (Wird z.B. ein Teil eines IP-Paketes nicht
        "ubertragen, braucht man den Rest auch nicht mehr).
      \subsubsection*{Jitter-Kontrolle}
        Manchmal ben"otigt man m"oglichst gleichbleibende Verz"ogerungszeiten.
        In diesem Fall k"onnten Route Pakete, die zu schnell sind verz"ogern,
        und zu langsame Pakete bevorzugt zustellen.
      \subsubsection*{Resource Reservation Protocol}
        Dies ist ein Protokoll zur Reservierung von Resourcen beim Multicast. Es
        wird (irgendwie) ein spannender Baum aufgebaut. Entlang dieses Baumes
        k"onnen jetzt Resourcen f"ur Multicast-Verbindungen reserviert werden,
        dabei wird jede Teilstreke logischerweise maximal f"ur eine "Ubertragung
        reserviert.
    \subsection*{Tunneling}
      Manchmal muss ein Paket "uber eine Strecker "ubertragen werden, die das
      entsprechende Protokoll nicht unterst"utzt. In diesem Fall k"onnen die
      Pakete in Pakete des fremden Protokolls \glqq eingepackt\grqq\ werden, und
      am Ende der Strecke (bzw. des Teilnetzes) werden sie wieder ausgepackt. Am
      Anfang und Ende der Tunnelingstrecke werden spezielle Router ben"otigt,
      die das Ein- und Auspacken erledigen.
    \subsection*{Fragmentierung}
      Manchmal unterst"utzt eine Teilstrecke nur eine maximale Paketgr"o"se, die
      kleiner ist, als das Paket. In diesem Fall mu"s das Paket von einem Router
      in kleinere Teile zerlegt (fragmentiert) werden. Das Zusammensetzen der 
      Fragmente erledigt entweder ein anderer Router (transparent) oder das
      Endsystem selber. Ein Fragment mu"s irgendwie die Information enthalten,
      zu welchem Paket es geh"ort, und an welche Stelle des Paketes. Das kann
      mit laufenden Nummern, oder einem Byteoffset zum Paketanfang geschehen.
      Problematisch wird es, wenn Pakete mehrfach fragmentiert werden, oder wenn
      ein Fragment verloren geht und wiederholt werden soll.
    \subsection*{IP - Internet Protocol}
      IP ist das Vermittlungsschicht-Protokoll im Internet. Es bietet einen
      verbindungslosen, unzuverl"assigen, paketorientierten Dienst an. Die
      Steuerprotokolle ICMP und (R)ARP unterst"utzen IP bei seiner Aufgabe.
      \subsubsection*{IP-Paketkopf}
        Der IP Paketkopf enth"alt die Version des verwendeten IP-Protokolls
        (4), die L"ange des Paketkopfes, eine (nicht verwendete)
        Servicetypebeschreibung, die Gesamtl"ange des
        Paketes, eine Eindeutige ID f"ur die Fragmentierung, einen
        Fragmentoffsett f"ur die Fragmentierung, die Bits DF (don't fragment)
        und MF (more fragments), die Restlebensdauer des Paketes, einen Code
        f"ur das verwendete Protokoll der Transportschicht, eine Pr"ufsummen
        f"ur den Paketkopf und jeweils eine 32 bit lange Ziel- und
        Quelleadresse die weltweit eindeutig ist. Au"serdem stehen 40
        Byte f"ur optionale Header zur Verf"ugung. Es sind
        Header-Optionen wie Source Routing oder Route Logging oder Security
        definiert, diese werden jedoch von den existierenden Routern nicht
        unterst"utzt.
      \subsubsection*{IP-Adressierungsschema}
        Der IP-Adressraum ist in mehrere Nezklassen eingeteilt. Ein IP-Adresse
        besteht aus einem Klassencode, der Netzadresse und der Hostadresse. Die
        Hostadresse kann eventuell noch in Subnetz- und Hostadresse aufgeteilt
        werden. Die urspr"ungliche Form sah 5 Klassen vor: A (Code: 0), 16 Netze
        mit 16 Millionen Hosts, B (01), 16k Netze mit 64k Hosts, C (011), 2
        Millionen Netze mit 256 Hosts, D (0111) f�r Multicast-Adressen und E
        (01111) f"ur sp"atere Verwendung. Es sind also zwei bzw. drei Schichten
        f"ur das Routing vorgesehen: Ein Paket wird zuerst zu seinem Netz
        geleitet und dort (vielleicht "uber Subnetze) intern weiterverteilt.
      \subsubsection*{CIDR}
        Leider waren nach dem urspr"unglichen Schema die Adressen irgendwann
        viel zu knapp. Viele Firmen reservierten ein Class B Netz, obwohl sie
        gar nicht so viele Adressen brauchten, denn die 256 Host der Klasse C
        waren zu wenig. Die Einrichtung von Millionen Teilnetzen (z.B. mehr
        Klasse C) w"urde au"serdem zu einer Explosion der Routingtabellen
        f"uhren. Um das Problem zu entsch"arfen, wurde die Vergabe von
        Class-C-Subnetzen ge"andert: Statt als Subnetze werden die Adressen
        jetzt in variabel gro"sen Bl"ocken vergeben. Um das Routing zu
        vereinfachen, wird der Adressraum in Zonen unterteilt (Nord- und
        S"udamerika, Europa und Asien) die dann f"ur das Routing als \glqq
        Subnetze \grqq\ gelten. Damit nicht jedes C-Netz in einem Block einzeln
        in die Routingtabelle mu"s, wird zus"atzlich zu einem Block eine
        Netzmaske vergeben. Die Netzmasken werden dann mit den eingehenden
        Paketen AND-verkn"upft um den richtigen Eintrag zu finden (d.h. es wird
        gepr"uft, ob der Anfang der Adresse mit dem Anfang des entsprechenden
        Routing-Eintrages "ubereinstimmt).
      \subsubsection*{IP-Routing}
        Das Internet besteht aus einer gro"sen Zahl von \textit{autonomen
        Systemen} (also z.B. Firmennezen). Der Betreiber eines solchen Systems
        kann das Routingverfahren dort selbst bestimmen (internes Routing). Das
        Routing zwischen den einzelnen autonomen Systemen wird als externes
        Routing bezeichnet.
      \subsubsection*{Open Shortest Path First}
        OSPF ist das empfohlene Routingprotokoll f"ur internes Routing, ein
        Link-State-Protokoll. (Fr"uher wurde das Distance-Vektor-Protokoll RIP
        verwendet, aber es hatte recht viele Schw"achen).
        
        OSPF sieht ein System als Ansammlung von \textit{Bereichen} an. Mehrere
        Bereiche "uberlappen sich nicht, es m"ussen aber nicht alle Hosts zu
        einem Bereich geh"oren. Alle Router sind auf jeden Fall an den
        Backbone-Bereich angeschlossen. Ein Router in einem Bereich berechnet
        immer nur die Routen f"ur diesen Bereich, ein Paket f"ur einen anderen
        Bereich wird an das Backbone gesandt, dort weiterverteilt und
        schlie"slich von den Routern im Zielbereich ausgelierfert. Ein Router
        mu"s also nicht immer die komplette Topologie des Netzes kennen.
        
        OSPF verwendet ein einfaches Protokoll zum Datenaustausch zwischen den
        Routern das auf IP aufsetzt. Die Nachrichten werden best"atigt. Ein
        Router kann HELLO-Pakete aussenden, um seine Nachbarn zu finden (per
        Multicast). Ein LINK STATE-Paket wird von einem Router regelm"a"sig oder
        bei Bedarf verschickt und zwar nicht an alle Nachbarn, sondern nur an
        sogenannte \textit{angrenzende} Router. Es gibt einen designierten
        Router, der an alle anderen angrenzt, ein Backup f"ur ihn steht
        jederzeit bereit. Router k"onnen au"serdem den Stand seiner
        Linkdatenbank aussenden, oder von einem anderen Link-State-Informationen
        einfordern.
        
        Anhand ihrer Informationen berechnen die Router dann die jeweils
        k"urzesten Pfade zu den einzelnen Zielen.
      \subsubsection*{Border Gateway Protocol}
        Dies ist das Protokoll f"ur das externe Routing im Internet. Beim
        externen Routing m"ussen gewisse Regeln beachtet werden, z.B. wollen
        eventuell gewisse Netze bestimmten Verkehr nicht "ubertragen. Das BGP
        ist ein deutlich verbessertes Distance-Vektor-Verfahren, da"s mit
        solchen Regeln umgehen kann. Ein Router speichert nie nur die Entfernung
        und den n"achsten Router zu einem Ziel, sondern den kompletten Pfad. Er
        erh"alt von den Nachbarn verschiedene Pfade zu den Zielen, und kann
        davon den besten aussuchen; Pfade die durch ihn selber f"uhren
        (vermeidet Count-To-Infinity) und Pfade die nicht zu den Regeln passen,
        k"onnen dabei verworfen werden. Hat der Router einen Pfad ausgew"ahlt,
        so "ubertr"agt er ihn an alle anderen als \glqq seinen \grqq\ Pfad.
      \subsubsection*{ICMP}
        Das Internet Control Message Protocol dient zu Wartungszwecken im
        Internet. Hier liegt z.B. der ECHO-Mechanismus (Pr"ufen von
        Verbindungen) und CHOKE-Pakete. Jede ICMP-Nachricht wird in ein IP-Paket
        verpackt.
      \subsubsection*{ARP/RARP}
        Das Adress Resolution Protocol dient der Zuordnung von IP-Adressen zu
        Schicht-2-Hardwareadressen. Es wird ein Broadcast gesendet, mit der
        Aufforderung da"s sich der Besitzer der IP-Adresse x bitte melden m"oge.
        Alle, die diesen Broadcast h"oren, k"onnen ihren ARP-Cache
        aktualisieren, und wissen jetzt unter welcher Hardwareadresse sie diese
        IP erreichen. 
        
        Beim RARP kann eine Station eine Anfrage rundsenden, welche
        IP-Adresse zu ihrer Hardwareadresse geh"ort. Diese Anfrage wird von
        einem designierten RARP-Server beantwortet, der die Zuordnungen
        gespeichert hat.
      \subsubsection*{Multicast-Routing}
        IP bietet Multicast-Dienst f"ur Gruppen an. Multicast-Adressen liegen im
        Bereich 224.x.y.z, manche dieser Adressen haben festgelegte Bedeutungen
        (z.B. 224.0.0.1 f"ur das lokale Netzwer). F"ur das Multicasting werden
        spezielle Multicast-Router ben"otigt (MBone), die regelm"a"sig die
        angeschlossenen Hosts nach ihrer Gruppenzugeh"origkeit fragen. Die
        Multicast-Router tauschen die Daten dann entlang eines spannenden Baumes
        aus.
      \subsubsection*{IPv6}
        Die n"achste Generation des IP-Protokolls soll vor allen Dingen den
        Mangel an Adressen beheben, au"serdem wurde der Header vereinfacht, um
        die Verarbeitung der Pakete effizienter gestalten zu k"onnen.
        
        Der \textbf{IPv6-Header} besteht aus dem "ublichen Feld mit der
        Versionsnummer (6), einem Flow-Control-Feld das zusammen mit speziellen
        Routern Datenfl"usse (d.h. Pseudo-Verbindungen) erm"oglichen soll, einer Angabe
        "uber die L"ange der Nutzlast (maximal 64k), einem
        Feld mit dem Typ des n"achsten (optinalen) Headers (falls kein Header
        mehr folgt, steht hier der Code des Schicht-4-Protokolls), und ein Feld mit der
        restlichen Lebensdauer in Hops. Weggefallen sind die Checksumme (mu"ste
        bei IPv4 in jedem Router neu berechnet werden) und die Mechanismen zur
        Fragmentierung: Wenn in IPv6 ein Router ein Paket nicht annehmen kann,
        lehnt er es ab und informiert den Sender "uber seine MTU (maximum
        transfer unit). Der Sender ist daf"ur verantwortlich, da"s die maximal
        zul"assige Paketgr"o"se nicht "uberschritten wird. Die Quell- 
        und Ziel-\textbf{Adressen} schlie"slich sind 128 bit lang und wieder
        hierarchisch gegliedert (nach Regionen, dann nach Providern, etc.). Die
        Adressen des bisherigen IPv4 sind auch im IPv6-Adressraum enthalten.
        
        IPv6 unterst"utztl eine Reihe von \textbf{optionalen Headern}, z.B. f"ur
        das Versenden von Jumbogrammen (Paketen > 64k), Routingfunktionen,
        Sicherheit, Verschl"usselung und "ahnliches.
  \section{Transportschicht}
    Die Transportschicht bietet zwei entfernten Kommunikationspartnen Dienste
    f"ur die Ende-zu-Ende-Kommunikation an, z.B. zuverl"assige virtuelle
    Verbindungen. Dies geschieht unabh"angig von den F"ahigkeiten der
    Vermittlungsschicht. 
    
    In gewisser Weise hat die Transportschicht "ahnliche Aufgaben wie die
    Sicherungsschicht: Sie mu"s die Unzul"anglichkeiten tieferer Schichten
    ausgleichen, um den Anwendungen einen zuverl"assigen Dienst zu bieten.
    Deshalb kommen auch Mechanismen aus der Sicherungsschicht wie die
    Flusskontrolle hier wieder zum Einsatz. Andererseits unterscheidet sich die
    Situation hier grundlegend von einer Direktverbindung: Pakete k"onnen
    dupliziert und verz"ogert werden, und nicht nur der Kommunikationspartner
    sondern auch das Netz selbst kann "uberlastet sein. Au"serdem hat eine
    Transportschicht-Instanz m"oglicherweise viele Verbindungen mit variabler
    Paketgr"o"se gleichzeitig zu versorgen.
    \subsection*{Flu"skontrolle}
      Grunds"atzlich kommt hier, wie in der Sicherungsschicht, ein
      Schiebefenster zum Einsatz. Man kann hier aber nicht davon ausgehen, da"s
      unbest"atigte Pakete auch wirklich verschwunden sind, sie k"onnte ja auch
      nur verz"ogert sein und zu einem ung"unstigen Zeitpunkt wieder auftauchen.
      Duplizierte Pakete (ein Duplikat tauch zum falschen Zeitpunkt noch mal
      auf) verschlimmern das Problem noch.
      Um dieses Problem zu l"osen, ben"otigt man einen Mechanismus der
      verz"ogerte Pakete (und ihre Best"atigungen) irgendwann verwirft -- es
      gibt also eine Zeitspanne $ T $, nach der ein Paket und alle
      Best"atigungen auf jeden Fall verschwunden sind (Das l"a"st sich mit
      Timeouts erreichen). Andererseits ist es notwendig, da"s ung"ultige Pakete
      erkannt und verworfen werden.
      
      Eine Methode w"are die Verwendung von eindeutigen Verbindungs-IDs damit
      die Partner herausfinden k"onnen, ob ein Paket zu einer g"ultigen
      Verbindung geh"ort. Allerdings m"ussten in diesem Fall die IDs aller alten
      Verbindungen gespeichert werden (und sie w"aren verloren, wenn der Host
      abst"urzt...). Eine andere M"oglichkeit ist die Verwendung von
      Zeitangaben: Dazu m"ussen die beiden Rechner synchronisierte Uhren haben
      (die auch nach einem Absturz weiterlaufen). Als erste Sequenznummer einer
      Verbindung wird dann die aktuelle Uhrzeit verwendet (oder ein Teil davon),
      durch diese Methode sind die Sequenznummern eindeutig festgelegt, und
      veraltete Aufbauw"unsche w"urden abgelehnt. Der Folgenraum sollte so gro"s
      sein, da"s wandernde Pakete auf jeden Fall verschwunden sind, bevor
      sich die Folgennummern wiederholen (ein Problem bei Gigabitnetzen!). 
      
      Ein andere Problem kann noch auftauchen, wenn ein Host abgest"urzt war,
      wieder aufsetzt und vor dem Absturz Folgennummern verwendet wurden, die
      nun g"ultig sind (d.h. Folgennummern, die um bis zu $ T $ gr"o"ser sind,
      als die aktuelle Uhrzeit) -- sollten diese Pakete noch auftauchen, werden
      sie als g"ultig akzeptiert. Deshalb m"ussen Pakete, die f"ur eine neu
      aufgesetzte Verbindung noch g"ultig w"aren, verboten werden. Das betrifft
      alle Sequenznummern in einer verbotenen Zone der Breie $ T $ oberhalb der
      aktuellen Zeit (Pakete mit noch gr"o"serer Uhrzeit w"aren verworfen, bis
      die Uhrzeit diesen Stand erreicht hat). Das bedeutet allerdings auch, da"s
      h"ochstens ein Paket pro Zeittakt gesendet werden darf (die Uhr m"u"s also
      sehr schnell laufen), und da"s die verbotene Zone auch erreicht wird, wenn
      sich der Sender sich ihr durch langsames Senden \glqq vom anderen Ende
      \grqq\ aus n"ahert (die Folgenummern wiederholen sich ja).
    \subsection*{Pufferverwaltung}
      Die Transportschicht mu"s eventuell einen Haufen Verbindungen verwalten
      und unterst"utzt m"oglicherweise Transportschicht-Pakete (TPDUs) mit
      unterschiedlicher Gr"o"se. Es ist also oft nicht m"oglich und sinnvoll,
      einen Puffer mit festgelegten Gr"o"sen einzurichten. Stattdessen sollte
      die Transportschicht den Puffer dynamisch verwalten und nat"urlich
      sinnvoll organisieren (verkettete Puffer, Kreispuffer, wasauchimmer...).
      Damit das funktioniert, wird dann ein \textbf{Schiebefenster mit
      dynamischer Gr"o"se} verwendet: Der Empf"anger teilt in der Best"atigung
      jeweils mit, wie viele Pakete noch gesendet werden d"urfen. Wartet der
      Sender, und es werden wieder Puffer frei, wird das dem Sender auch
      mitgeteilt. Diese Mitteilung sollte nach einer Zeit wiederholt werden,
      falls sie verlorenging und der Sender immer noch wartet.
    \subsection*{"Uberlastungssteuerung}
      Erh"alt Schicht 3 von Schicht 4 konstant mehr Daten als sie "ubertragen
      kann wird sie irgendwann beginnen, Pakete zu verwerfen. In diesem Fall
      kann nur eine Drosselung der Transportschicht-Senderate wirklich eine
      Abhilfe schaffen. Deshalb mu"s der Sender sein Sendefenster nicht nur an
      den Empf"angern anpassen, sondern auch an die Gegebenheiten im Netz.
      Welche Mechanismen dort verwendet werden, wird am Beispiel von TCP
      erl"autert.
    \subsection*{Verbindungsauf- und Abbau}
      Damit beim Verbindungsaufbau nichts schiefgeht, wird ein
      Drei-Wege-Handshake verwendet: Verbindungsanfrage, Best"atigung,
      R"uckbest"atigung (mit erstem Datenpaket). G"abe es die R"uckbest"atigung
      nicht, k"onnte eventuell der zweite Partner zu senden beginnen obwohl die
      Best"atigung verlorgen gegangen ist. Auf der Transportschicht werden beim
      Verbindungsaufbau auch die intialen Sequenznummern ausgehandelt.
      
      Beim Verbindungsabbau kann wieder ein Dreiwege-Protokoll verwendet werden, obwohl
      hier nicht alle Fehlerf"alle auszuschlie"sen sind (byzantinisches
      Problem).
    \subsection*{TCP}
      TCP ist das Protokoll der Transportschicht im Internet. Es stellt
      zuverl"assige Punkt-zu-Punkt-Verbindungen zwischen zwei Applikationen zur
      Verf"ugung. Diese Verbindungen sind immer Vollduplex-Verbindungen, und
      "ubertragen Bytestr"ome. TCP "ubertr"agt die Daten in sogenannten
      Segmenten, wobei die Aufteilung des Bytestromes in Segmente von TCP
      v"ollig willk"urlich vorgenommen werden kann.
      \subsubsection*{Berkeley Sockets}
        Ein Socket ist ein Zugangspunkt zur Transportschicht. Mann kann ein
        Socket erzeugen, ihm eine Adresse zuweisen, Verbindungen auf ihm
        entgenenehmen oder eine ausgehende Verbindung aufbauen.
        Selbstverst"andlich k"onnen durch das Socket dann auch Daten gesendet
        und empfangen werden.
      \subsubsection*{TCP-Adressierungsschema}
        Eine TCP-Verbindung ist eindeutig gekennzeichnet durch die Adressen der
        beiden Endpunkte. Eine TCP-Adresse besteht aus der IP-Adresse und dem
        \textbf{Port}. Die Ports unter 256 sind wohlbekannt (\textbf{well
        known}) und Standarddiensten zugeordnet (z.B. Mail, Telnet, etc.)
        H"ohere Portnummern ("uber 1024) k"onnen von Applikationen frei 
        verwendet werden, z.B. zum Aufbau von Verbindungen.
      \subsubsection*{TCP-Semgmentheader}
        Der TCP-Segmentheader enth"alt zun"achst einmal die Ziel- und
        Quellportnummer der Verbindung. Au"serdem ist eine 32-Bit Sequenznummer
        und ein Best"atigungsnummer enthalten (die Best"atigungsnummer
        entspricht der n"achsten erwarteten Sequenznummer). Ein weiteres Feld
        beschreibt das Empfangsfenster des Senders (d.h. wie viele Byte ab dem
        best"atigten noch gesendet werden d"urfen). Ein Feld gibt die L"ange des
        TCP-Headers an (es sind optionale Felder m"oglich) und eine Pr"ufsumme
        sichert den TCP-Header, die Daten und den sog. Pseudo-Header der die
        IP-Adressen von Sender und Empf"anger enth"alt. Das ACK-Bit gibt an,
        da"s die Best"atigung in diesem Segmentheader g"ultig ist. Das SYN-Bit
        dient dem Verbindungsaufbau. Das PSH-(Push-)Bit gibt an da"s Push-Daten
        gesendet werden (die sofort an die Anwendung ausgeliefert werden
        sollen). Das URG-(Urgent-)Bit zeigt dringende Daten an, die sofort
        ausgeliefert werden (eine Art Interrupt-Signalisierung) in diesem Fall
        verweist ein Urgent-Zeiger im Header auf den Beginn der dringenden
        Daten. Das RST-(Reset-)Bit dient dem Zur"ucksetzen der Verbindung nach
        einer St"orung, das FIN-Bit zum Verbindungsabbau.
        
        Nach dem eigentlichen Segmentheader k"onnen noch weitere Optionen
        folgen. Tats"achlich verwendete Optionen sind \textbf{Window Size}, das
        eine andere Skalierung der Fenstergr"o"se vorgibt um ein Empfangsfenster
        von > 64k zu erm"oglichen und die implementierung der selektiven
        Wiederholung.
      \subsubsection*{Verbindungsauf- und abbau}
        Eine TCP-Verbindung wird initiert, indem ein Proze"s (A) eine
        Verbindungsanfrage an einen warten Proze"s (B) (der ein Socket "uberwacht)
        stellt. Dazu sendet Proze"s A ein SYN-Segment mit seiner initialen
        Sequenznummer. B antwortet mit einem SYN-Paket, da"s die Anfrage
        best"atigt und Bs initiale Segmentnummer enth"alt (SYN ACK). Daraufhin
        best"atigt A wiederum das eingegangene Paket (ACK), die Verbindung ist
        aufgebaut und beide Partner k"onnen Daten senden.
        
        Die beiden Richtungen einer TCP-Verbindung werden getrennt abgebaut:
        Jede Seite sendet ein FIN-Segment, wenn sie ihre "Ubertragung beendet hat,
        und wartet auf die best"atigung. Wenn eine Station ein FIN-Segment
        erh"alt, kann sie mit der Best"atigung auch gleich den Abbauwunsch der
        anderen Richtung "ubermitteln (FIN ACK).
      \subsubsection*{Nagles Algorithmus und Silly Window Syndrome}
        Das Lesen und Schreiben von einzelnen Bites kann in TCP zu Problemen
        f"uhren. Senden Stationen immer nur einzelne Bytes, so wird f"ur jedes
        Byte ein eigenes, 21 Bit gro"ses Segment erzeugt und unn"otig Bandbreite
        verschwendet. Nagles Algorithmus sieht daher vor da"s, wenn ein
        einzelnes Byte "ubertragen wird, die "Ubertragung angehalten wird bis
        die Best"atigung eintrifft (in der Hoffnung, da"s sich noch ein paar
        Bytes mehr ansammeln.
        
        Umgekehrt kann es vorkommen, da"s ein Proze"s immer nur einzelne Bytes
        aus dem Empfangpuffer liest. In diesem Fall w"urde der Empf"anger jedes
        Mal ein neues Empfangsfenster von einem Byte signalisieren, und der
        Empf"anger jeweils ein einzelnes Byte "ubertragen. Dies ist das
        Silly-Window-Syndrom und erzeugt ebenfalls sinnlosen Overhead. Zur
        Abhilfe mu"s der Empf"anger solange warten, bis er wieder ein Fenster
        mit einer sinnvollen Gr"o"se signalisieren kann.
      \subsubsection*{"Uberlastungs"uberwachung}
        TCP verwaltet f"ur jedes Segment einen Timeout, nachdem es erneut
        "ubertragen wird. Es wird davon ausgegangen, da"s eine ausbleibende
        Best"atigung auf eine "Uberlastung im Netz zur"uckgeht (eine Annahme,
        die f"ur Kabel- bzw. insbesonder f"ur Glasfaser"ubertragung gemacht
        werden kann, bei Funk"ubertragung gibt es hier Probleme). TCP verwaltet
        zus"atzlich zum Empfangsfenster ein sogenanntes "Uberlastungsfenster
        (Congestion Window),
        das ebenfalls angibt wie viele Bytes der Sender "ubertragen kann. 
        
        Das "Uberlastungsfenster wird in TCP durch den \textbf{Slow Start}
        Algorithmus angepasst: Das Fenster hat zuerst einen Minimalwert (z.B.
        eine maximale Segmentgr"o"se) und wenn alle Daten im Segment erfolgreich
        "ubertragen wurden, wird das Fenster verdoppelt. Dies geht solange, bis
        entweder Daten verlorengehen, oder die Gr"o"se des Empfangsfenster
        erreicht wurde. Zus"atzlich wird ein Schwellenwert gef"uhrt, der bei
        Paketverlusten auf die Gr"o"se des halben "Uberlastungsfensters gesetzt
        wird. Bis der Schwellenwert erreicht wird, verdoppelt der
        Algorithmus das "Uberlastungsfenster, dar"uber wird linear
        weitervergr"o"sert.
      \subsubsection*{Timermanagement}
        Ein Problem im TCP ist die Verwaltung des Timeouts f"ur verlorene Pakete
        -- einfach weil nicht genau bekannt ist wann die Best"atigung eigentlich
        ankommt. TCP behilft sich, indem es versucht, zuerst einmal die
        durchschnittliche Rundreisezeit bis zum Ziel zu bestimmen:
        \[
        	R = \alpha R + (1 - \alpha)M
        \]
        Dabei ist $ R $ die gesch"atzte Rundreisezeit und $ M $ die gemessene
        Zeit von Senden bis zur Best"atigung des letzten Paketes. $ \alpha $ ist
        ein Gl"attungsfaktor. Zus"atzlich wird versucht, ann"ahernd die
        Standardabweichung der Best"atigungsankunft zu bestimmen:
        \[
        	D = \alpha D + (1 - \alpha) | RTT - M |
        \]
        Der Gl"attungsfaktor $ \alpha $ braucht nicht unbedingt der selbe zu
        sein wie oben. $ D $ ist zwar nicht die Standardabweichung aber eine
        hinreichend sinnvolle N"aherung. Der Timeout $ T $ berechnet sich dann
        nach 
        \[
        	T = R + 4D
        \]
        also ein Paket wird nach der durchschnittlichen Rundreisezeit plus der
        vierfachen Standardabweichung als verloren angesehen. Das ist relativ
        konservativ: Etwas zu lange warten ist besser als st"andig neu zu
        "ubertragen und Slow Start anzuwenden.
    \subsection*{TCP in mobilen Umgebungen}
      Beim mobilen Einsatz macht sich vor allem Slow Start unangenehm bemerkbar:
      Funkverbindungen verlieren recht h"aufig Pakete, und eine sinnvolle
      Strategie w"are eigentlich, diese sofort neu zu "ubertragen. Andererseits
      w"urde ohne Slow Start das Internet zusammenbrechen, der Mechanismus
      sollte also weiterhin in IP implementiert bleiben.
      \subsubsection*{Fast Retransmit/Fast Recovery}
        Diese TCP-Option geht davon aus, da"s keine "Uberlastung vorliegt
        solange noch \textit{irgendwelche} Best"atigungen von Empf"anger
        eingehen (selbst wenn es nicht die f"ur das aktuelle Paket sind). In
        diesem Fall versucht TCP die nicht best"atigten Segmente sofort noch
        einmal zu "ubertragen (Fast Retransmit). Au"serdem liegt kein Grund vor,
        Slow Start anzuwenden und die Datenrate wird beibehalten (Fast
        Recovery). Slow Start wird nur angewendet, wenn die Best"atigungen ganz
        ausbleiben.
        
        Eine weitere M"oglichkeit ist es, da"s der mobile Knoten diesen
        Mechanismus erzwingt, indem er von sich aus Best"atigungsduplikate
        versendet.
      \subsubsection*{Indirect TCP}
        Bei dieser Variante dient der Zugangspunkt (bzw. Foreign Agent)
        des mobilen Host gleichzeitig
        als TCP-Proxy. Er best"atigt alle Segmente des Kommunikationspartners,
        puffert sie und liefert sie dann selbst "uber eine TCP-Verbindung an den
        Knoten aus. Am Knoten selbst und am Kommunikationspartner mu"s nichts
        ge"andert werden, und das Verfahren ist recht effizient. Alllerdings
        geht die Ende-Zu-Ende-Semantik von TCP verloren, kann der
        Knoten nicht erreicht werden, werden die Segmente nicht ausgeliefert
        \textit{obwohl sie schon best"atigt sind}. Au"serdem m"ussen beim Wechsel
        des Zugangspunktes alle gepufferten Daten zum neuen Zugangspunkt
        "ubertragen werden.
      \subsubsection*{Snooping TCP}
        Hier trennt der Zugangspunkt die Ende-zu-Ende-Verbindung nicht,
        puffert allerdings alle Pakete und versucht sie schnell zu wiederholen,
        Pakete die im Datenstrom des Mobilknotens fehlen werden direkt
        nachgefordert und "ahnliches. Dies hat den Vorteil, da"s die Verbindung
        wirklich bis zur Mobilstation geht, allerdings wird das Netz nicht so
        gut gegen die Fehler der Funkstrecke abgeschirmt.
      \subsubsection*{Mobile TCP}
        Bei Mobile TCP trennt der Zugangspunkt die TCP-Verbindung zwar auch,
        puffert aber keine Daten und sendet auch keine Best"atigungen. (Der
        Zugangspunkt geht davon aus, da"s die Qualit"at der Funkstreck o.k. ist,
        also sollen sich die Partner selbst um Wiederholungen k"ummern).
        Stattdessen untersucht der Zugangspunkt, ob noch Best"atigungen vom
        Mobilen Knoten kommen. Ist das nicht der Fall geht er davon aus, da"s
        der mobile Host nicht erreichbar ist und setzt das Empfangsfenster auf
        null. Daher wird der Kommunikationspatner nicht versuchen, in dieser
        Zeit Daten zu senden.
  \section{Sitzungsschicht}
    Diese Schicht sollte es Benutzern erm"oglichen, Sitzungen zwischen zwei
    Maschinen aufzubauen. Dies geschieht allerdings in der Praxis auf
    Anwendungsebene, und nicht im Protokollstapel.
  \section{Darstellungsschicht}
    Diese Schicht kennt die Repr"asentation von Daten und repr"asentiert sie
    auf standartiesierte Weise. Z.b. k"onnten Datenformate entsprechend
    konvertiert werden. Diese Schicht wird praktisch nie eingesetzt,
    Konvertierungen finden auf Anwendungsebene statt.
  \section{Verarbeitungsschicht}
    Diese Schicht enth"alt die Anwendungsspezifischen Protokolle.
  \section{Sonstige Systeme}
    \subsection*{ISDN}
      ISDN stellt verschiedene Digitale Dienste f"ur Telefonie und
      Daten"ubertragung zur Verf"ugung. ISDN definiert eine Reihe von
      "Ubertragungskan"alen. Benutzt werden allgemein der B-Kanal, ein 64 kbps
      Daten-/Sprachkanal (Ende-zu-Ende) der eine \glqq Datenpipeline\grqq\ 
      auf Schicht 1
      darstellt (f"ur Sprache wird er als PCM-Kanal benutzt), und der D-Kanal
      f"ur Zeichengabe, der 16 kbps (paketweise) gesichert "uber ein HDLC/LAPD-"ahnliches
      Protokoll "ubertr"agt (Schicht 3).
      
      ISDN definiert eine Reihe von Standardschnittstellen: V, f"ur die
      "Ubertragung zwischen Vermitllungstelle und Vermittlungsanschlu"s (?), U
      zwischen Vermittlungsstelle und Netzwerkterminator (NT, entspr. NTBA), und
      T nach dem Basisanschlu"s. Entweder werden an die T-Schnittstelle
      ($ S_0 $-Bus) bis zu 8 Ger"ate gleich angeschlossen, oder es gibt eine
      Nebenstellenanlage, die dann eine S-Schnittstelle (auch $S_0$-Bus) zur
      Verf"ugung stellt. "Uber einen Terminaladapter (TA, entspr. a/d-Wandler)
      k"onnen nicht ISDN-f"ahige Ger"ate an die R-Schnittstelle angeschlossen
      werden.
      
      Die D-Kanal-Rahmen im ISDN enthalten eine SAPI (Service Access Point
      Identifier, gibt den Dienst an (so "ahnlich wie eine Portnummer), und eine
      TEI (Terminal Endpoint Identifier), die (0-126) das ISDN-Ger"at anspricht
      (127 -> Rundruf). Die TEI kann fest eingestellt sein, oder wird von der
      Vermittlungsstelle vergeben.
      
      Beim ISDN werden die Kan"ale von der Vermittlungsstelle bis zum NT
      (U-Schnittstelle) "uber eine gew"ohnliche Telefonleitung gemultiplext. Es
      werden jeweils 48 bit in 250 $\mu$s "ubertragen, d.h. 192 kbit/s. Jeder
      dieser Rahmen enth"alt 12 bit an Protokollinformationen, so da"s 144 
      kbit/s f"ur zwei B-Kan"ale und den D-Kanal bleibt.
    \subsection*{GSM}
    \subsection*{Bluetooth}
    \subsection*{Frame Relay}
    \subsection*{X25}
    \subsection*{ATM}
      ATM wird absichtlich nicht innerhalb der "ublichen sieben Schichten
      behandelt, da ATM ein eigenes Referenzmodell hat. Zwar wird offensichtlich
      hei"s diskutiert, welche ATM-Schicht jetzt welcher OSI-Schicht entspricht,
      doch das ist wohl eher eine Glaubensfrage. Wir nehmen das ATM Modell als
      gegeben hin.
      
      Eigentlich ist das ATM-Referenzmodell kein reines Schichtenmodell, sondern
      wird oft als W"urfel dargestellt. Zus"atzlich zu den Schichten Besitzt ATM
      mehrere \textit{Ebenen}, und zwar eine Ebene f"ur das Ebenenmanagement,
      und eine f"ur das Schichtenmanagement (die nat"urlich f"ur jede Schicht
      vorhanden ist). Die eigentlichen Schichten sind nochmals in eine Netz- und
      eine Benutzerschicht geteilt.
      
      Die eigentlichen \textbf{ATM-Schichten} sind die physikalische Schicht,
      die ATM-Schicht und die ATM-Anpassungsschicht.
      
      Die ATM-Schicht ist der eigentliche Kern der Hierarchie, sie stellt einen
      reihenfolgentreuen aber unzuverl"assigen Dienst zur Verf"ugung, der 53
      Byte lange \textbf{ATM-Zellen} "uber virtuelle Verbindungen transportiert.
      \subsubsection*{Physikalische Schicht}
        Die physikalische Schicht in ATM ist zweigeteilt. Der untere Teil, die
        eigentliche physikalische Schicht, ist f"ur die eigentliche "Ubertragung
        der Daten "uber das Medium zust"andig.
        
        Der obere Teil der Schicht, der Transmission Convergence (TC) Layer,
        "ubernimmt Aufgaben, die im OSI-Modell in Schicht 2 liegen:
        Zellenbildung, Berechnung der Zellchecksumme und "ahnliches mehr. Beim
        einem synchronen Medium werden zus"atzlich leeren (tr"age) Zellen in den
        Strom eingef"ugt wenn keine Daten vorliegen. Bei einem asychronen Medium
        ist das Verfahren zum Finden der Zellgrenzen recht witzig: Die Daten
        werden Bit f"ur Bit durch ein Schieberegister geschoben, bis ATM etwas
        findet das wie ein Zellkopf mit einer g"ultigen Pr"ufsumme aussieht. 
      \subsubsection*{Das ATM-Zellformat}
        Das Format der ATM-Zellen ist sowohl der ATM- als auch der
        physikalischen Schicht bekannt, was nat"urlich die Methoden des
        Schichtenentwurfes verletzt (allerdings scheint das bei praktischen
        Systemen h"aufiger vorzukommen... siehe TCP).
        
        Die ATM-Zellen kommen in zwei Versionen: User-zu-Netz (User Network
        Interface UNI) und Netz-zu-Netz (Network Network Interface NNI). Der
        einzige Unterschied zwischen den beiden ist, da"s die UNI-Zellen am
        Anfang ein zus"atzliches Feld haben (daf"ur eine k"urzer Path ID).
        Dieses Feld war f"ur irgendwelche Steurzwecke vorgesehen, und wird nicht
        benutzt. Jede Zelle hat zun"achst einmal eine Virtual Path ID (VPI) und
        eine Virtual Channel ID (VCI), die f"ur Routing und Adressierung genutzt
        werden. Au"serdem ist ein Feld f"ur mit Informationen "uber die
        enthaltenen Daten vorhanden (Payload Type, PTI), und ein Bit, da"s ob
        die Daten bei Staus als erste verworfen werden sollen (Cell Loss
        Priority, CLP). Der 40 Bit gro"se Header wird durch eine 8 Bit
        Pr"ufsumme gesch"utzt. Danach folgen 48 Byte ungesch"utzte Daten.
      \subsubsection*{Die ATM-Schicht}
        Die ATM-Schicht entspricht in vielen Belangen der Vermittlungsschicht im
        OSI-Modell. Insbesondere wird hier das Routing und die Stau"uberwachung
        realisiert. Die ATM-Schicht bietet den Endstellen einen
        reihenfolgentreuen, nicht zuverl"assigen Dienst an (d.h. Zellen d"urfen
        verloren gehen). "Uber die Dienstg"ute darf von den Endstellen und vom
        Netz verhandelt werden.
        
        Beim \textbf{Verbindungsaufbau} kann zun"achst einmal zwischen
        verschiedenen \textbf{Dienstklassen} gew"ahlt werden:
        \begin{itemize}
          \item{\textbf{Constant Bitrate, CBR:} Es wird eine feste Bandbreite
          f"ur diese Verbindung reserviert. Diese Brandbeite steht immer zur
          Verf"ugung.}
          \item{\textbf{Variable Bitrate, VBR:} Dieser Dienst ist f"ur
          Anwendungen mit variablen Bitraten ausgelegt, er ist in einer Version
          mit Echtzeitanforderungen (VBR-RT) ohne Jitter und in einer Version
          ohne Echtzeitanforderungen (VBR-NRT) zu haben. Auch hier werden die
          Resourcen im Vorraus reserviert}
          \item{\textbf{Available Bitrate, ABR:} Hier werden eine minimale und
          eine maximale Bitrate ausgehandelt. Die Resourcen f"ur die minimale
          Bitrate werden reserviert, alles was dar"uber hinausgeht wird nach
          bestem Bem"uhen zur Verf"ugung gestellt. Dies ist die einzige Klasse,
          die Informationen "uber "Uberlastungen erh"alt.}
          \item{\textbf{Unspecified Bitrate, UBR:} Grob gesagt: Die Station darf
          Daten senden wie sie will und wenn Resourcen vorhanden sind werden
          die Daten sogar "ubertragen. Eine Benarichtigung "uber "Uberlastung
          findet nicht statt.}
        \end{itemize}
        ATM benutzt in jedem Fall feste Verbindungen, die in alle Router entlang
        der Strecke eingetragen werden -- Zellen der gleichen Verbindung laufen
        also auch immer "uber die gleiche Route. Welche Route eine Zelle
        einschlagen soll, entscheidet der Router anhand der Path ID, die Channel
        ID dient dazu, verschiedene Verbindungen zwischen den gleichen
        Endsystemen zu unterscheiden (allerdings ist es theorestisch auch
        m"oglich, da"s die Router auch die Channel ID heranziehen). Um einen
        solchen Pfad aufzubauen, wird zuerst eine Verbindungsanfrage (SETUP) 
        "uber Pfad 0, Kanal 5 an die jeweils n"achste Gegenstelle gesendet. Die
        Gegenstelle best"atigt die Verbindungsanfrage (CALL PROCEEDING),
        reserviert die ben"otigten Resourcen, tr"agt die Verbindung in seine
        Routingtabelle ein und kontaktiert den n"achsten
        Router auf dem Pfad. Sind nicht genug Resourcen vorhanden, wird der
        Verbindungswunsch abgelehnt und es kann versucht werden einen anderen
        Pfad zu verwenden. Dies geht solange weiter, bis der gew"unschte
        Teilnehmer erreicht ist. Dieser sendet dann eine CONNECT-Nachricht auf
        dem selben Weg zur"uckt (auch diese wird wieder auf jeder Teilstreck
        best"atigt. Danach steht die Verbindung, und ein Router mu"s ein
        eingehendes Paket nur noch noch der Pfad-ID "uberpr"ufen und
        entsprechend weiterleiten. Um beim Verbindungsaufbau den richtigen Pfad
        in Richtung Ziel zu finden, k"onnen die Router einen beliebigen
        Routingalgorithmus verwenden. Verbindungen k"onnen au"serdem vom
        Netzbetreiber permanent eingerichtet werden. Um eine Verbindung
        abzubauen, sendet der Teilnehmer eine RELEASE-Nachricht, die von der
        Gegenstelle best"atigt wird. Diese Nachricht pflanzt sich dann wie
        gehabt durch das Netz fort, bis die Verbindung komplett abgebaut ist.
        
        Bei jedem Verbindungswunsch wird die \textbf{Dienstqualit"at} mit
        ausgehandelt, also z.B. die minimale, maximale und durchschnittliche
        Zellrate, zul"assige Verz"ogerung und Jitter, maximale Verlustrate und
        Toleranzen. 
        
        Eine \textbf{ATM-Adresse} kann verschiedene Formate haben, zul"assig
        sind z.B. OSI-Adresse und ISDN-Telefonnummern.
      \subsubsection*{Traffic-Shaping}
        Damit eine Station die ihr zugewiesene Bandbreite nicht "ubersteigt wird
        eine Art Bucket-Mechanismus verwendet: ATM geht davon aus, da"s
        h"ochsten alle $ T $ Zeiteinheiten eine Zelle gesendet wird. Eine
        gewisse Toleranz ist zul"assig, doch die n"achste Zelle ist auf jeden
        Fall erst zum Zeitpunkt $ 2T $ zul"assig, damit die Toleranz nicht
        ausgenutzt wird. 
      \subsubsection*{"Uberlastungs"uberwachung der ATM-Schicht}
        Eine "Uberlastungs"uberwachung f"ur das Netz wird nur in der
        ABR-Dienstklasse geboten. Dazu wird alle $ k $ Zellen eine
        Resourcenmanagement- (RC-) Zelle gesendet, die nach ihrer Ankunft
        umkehrt und in die Gegenrichtung zur"uckgesendet wird. Sowohl auf dem
        Hin- als auch auf dem R�ckweg k"onnen hier die Vermittler die maximal
        zul"assige Bitrate eintragen (Vermittler k"onnen auch selbst RC-Zellen
        losschicken). Kommt die RC-Zelle nicht zur"uck, wird dies als
        "Uberlastung interpretiert. 
      \subsubsection*{Die ATM-Anpassungsschicht (AAL)}
        Die AAL-Schicht bietet etwas praktischere Dienste an, insbesondere zum
        Verschicken gr"o"serer Datenpakete. Es gibt verschiedene Dienste,
        zuverl"assige und unzuverl"assige, aber keinen einfachen
        verbindungsorientierten Dienst wie TCP.
        
        Eigentlich besteht die AAL- Schicht aus drei Unterschichten: Einer
        AAL-Konvergenzschicht, die in einen allgemeinen und einen
        anwendungsspezifischen Teil geteilt ist, und der SAR-(Segmentation and
        Reassembly-)Teilschicht. 
        
        \textbf{AAL 1} bietet einen unzuverl"assigen Dienst f"ur z.B.
        Video"ubertragungen, der lediglich fehlende und falsch eingef"ugte
        Zellen erkennt. AAL 1 benutzt keinen Header auf der Konvergenzschicht
        und f"ugt zu jeder Zelle eine 3-bit Folgenummer hinzu die durch eine
        3-bit Checksumme und ein Parit"atsbit gesch"utzt wird. Optional ist noch
        ein Zeiger auf die Daten vorhanden, falls diese nicht an einer
        Zellgrenze ausgerichtet sind.
        
        \textbf{AAL 2} sollte einen "ahnlichen Dienst wie AAL 1 anbieten,
        allerdings besser auf variable Bitraten ausgerichtet. Der Standard wurde
        so eingerichtet, da"s er nicht zu benutzen ist (Feldl"angen fehlen).
        
        \textbf{AAL 3/4} bietet die zuverl"assige und unzuverl"assige
        "Ubertragung von Datenstr"omen oder Nachrichten bis 64k an. Au"serdem
        kann AAL 3/4 mehrere Sitzungen "uber einen einzelnen virtuellen Kanal
        multiplexen. Diese AAL-Schicht verwendet einen Header sowohl auf der
        Konvergenzschicht, als auch f"ur jede einzelne Zelle. Der
        Konvergenzschicht-Header enth"alt ein Statusfeld (Nachrichtentyp u."a.),
        ein Start- und Endeflag und eine L"angenangabe f"ur die enthaltenen
        Daten. Eine L"angenangabe wird auch noch noch dem Endeflag angeh"angt.
        Jede Zelle enth"alt dann noch einmal Typinformationen, einen
        Folgenummer, eine ID f"ur das Multiplexing und eine 10-Bit Checksumme
        (plus ein LI-Feld). 
        
        \textbf{AAL 5} wurde entwickelt, weil einigen Leuten der hohe Overhead
        bei AAL 3/4 nicht gefiel. AAL 5 bietet "ahnliche Dienste an, hat aber
        lediglich einen Header auf der Konvergenz-Schicht. Ein Paket besteht
        hier aus bis zu 64k Daten, einem Feld f"ur Steuerinformationen der
        h"oheren Schicht, einem Feld f"ur sp"atere Verwendung, einer
        L"angenangabe und einer 4-Byte Pr"ufsumme. In jeder Zelle werden alle 48
        Byte f"ur Daten benutzt, Paketgrenzen werden "uber ein Bit in den
        Steuerinformationen der ATM-Zelle erkannt. (Nicht ganz elegant, aber
        effizient.
  \section*{Copyright}
    Das Urheberrecht an diesem Dokument (Werk) liegt bei Daniel Hahn (Autor)
    (dhahn@gmx.de). 
    
    Dieses Werk darf - zu nichtgewerblichen Zwecken - 
    in elektronischer und gedruckter Form frei vervielf"altigt und 
    verbreitet werden.
    
    Die Erstellung und Verbreitung von Bearbeitungen, die auf diesem Werk
    basieren, ist - zu nichtgewerblichen Zwecken - gestattet wenn 
    \begin{itemize}
    	\item{die Bearbeitung als solche deutlich kenntlich gemacht wird.}
        \item{alle Hinweise auf den Autor, die Warnungen am Anfang sowie dieser
              Copyrighthinweis erhalten bleiben.}
        \item{der Autor der Bearbeitung allen Nutzern mindestens dieselben
              Rechte einr"aumt wie der Autor des urspr"unglichen Werkes.}
    \end{itemize}
    Jede sonstige Vervielf"altigung und Verbreitung des Werkes, 
    insbesondere die
    Vervielf"altigung zu gewerblichen Zwecken, bedarf der ausdr"ucklichen
    Genehmigung des Autors.
\end{document}
